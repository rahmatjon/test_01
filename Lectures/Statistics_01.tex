\documentclass[aspectratio=169]{beamer}
%\documentclass{beamer}
\usepackage[utf8]{inputenc}
\usepackage[T2A]{fontenc}
\usepackage[russian]{babel}
\usepackage{cmap}
\usetheme{Boadilla}

\newcommand{\Ki}{\mathscr{K}}
\newcommand{\D}{\mathscr{D}}
\newcommand{\be}{\begin{equation}}
\newcommand{\ee}{\end{equation}}
\newcommand{\bes}{\begin{equation*}}
\newcommand{\ees}{\end{equation*}}

\makeatletter
\setbeamertemplate{footline}{%
  \leavevmode%
  \hbox{%
    \begin{beamercolorbox}[wd=.92\paperwidth,ht=2.25ex,dp=1ex,center]{title in head/foot}%
      \usebeamerfont{title in head/foot}\insertshorttitle
    \end{beamercolorbox}%
  }%
  \begin{beamercolorbox}[wd=.08\paperwidth,ht=2.25ex,dp=1ex,right]{date in head/foot}%
    \usebeamerfont{date in head/foot}%
    \usebeamertemplate{page number in head/foot}%
    \hspace*{2ex} 
  \end{beamercolorbox}
  \vskip0pt%
}
\makeatother

\title{\bf Урок 1. Введение в статистику и основные понятия}
\author{{\bf Хакимов Р.И. + ChatGPT}}
 \date[\today]{}

\begin{document}
\begin{frame}
\titlepage
\end{frame}

\begin{frame}
\frametitle{Что такое статистика?}
\textbf{Статистика} -- это наука, которая занимается сбором, обработкой, анализом и интерпретацией данных. Она применяется во многих сферах жизни, включая экономику, медицину, социологию, бизнес, науку и многие другие..
\end{frame}
%------------------------------------------------

\begin{frame}
\frametitle{Значение статистики в современном мире. Примеры использования}
Статистика играет ключевую роль в современном мире, оказывая влияние на самые разные аспекты жизни. Вот несколько примеров ее использования:
\newline\\
\textbf{Экономика и бизнес}
\newline\\
\textbf{Анализ рынка:} Компании используют статистические методы для анализа рынка, определения целевой аудитории, оценки спроса и предложения, а также для прогнозирования продаж. Например, статистические данные могут помочь понять, какие продукты пользуются наибольшим спросом в определенных регионах.
\newline\\
\textbf{Оценка рисков:} В финансовом секторе статистика помогает оценивать риски инвестиций, разрабатывать стратегии управления портфелем и прогнозировать экономические тенденции.
\end{frame}
%------------------------------------------------

\begin{frame}
\frametitle{Значение статистики в современном мире. Примеры использования}
\textbf{Медицина и здравоохранение}
\newline\\
\textbf{Эпидемиология:} Статистика используется для анализа распространения заболеваний, оценки эффективности вакцин и лекарств, а также для прогнозирования возможных вспышек инфекций. Например, во время пандемии COVID-19 статистические модели использовались для прогнозирования распространения вируса и планирования мер по его сдерживанию.
\newline\\
\textbf{Клинические исследования:} Статистические методы применяются для оценки эффективности новых медицинских препаратов и методов лечения на основе данных клинических испытаний.
\end{frame}
%------------------------------------------------

\begin{frame}
\frametitle{Значение статистики в современном мире. Примеры использования}
\textbf{Наука и исследования}
\newline\\
\textbf{Анализ данных:} В науке статистические методы используются для анализа экспериментальных данных, выявления закономерностей и проверки гипотез. Например, в генетике статистика помогает в анализе генетических данных для понимания наследственных заболеваний.
\newline\\
\textbf{Метеорология:} Статистика играет важную роль в прогнозировании погоды, анализе климатических изменений и моделировании будущих климатических условий.
\end{frame}
%------------------------------------------------

\begin{frame}
\frametitle{Значение статистики в современном мире. Примеры использования}
\textbf{Спорт}
\newline\\
\textbf{Анализ эффективности игроков:} В спортивной аналитике статистика используется для оценки эффективности игроков и команд, разработки стратегий игры и прогнозирования результатов матчей.
\newline\\
\textbf{Фэнтези-спорт:} Статистические данные о производительности игроков используются в фэнтези-спорте для выбора команд и прогнозирования их успеха.
\end{frame}
%------------------------------------------------

\begin{frame}
\frametitle{Основные методы сбора данных}
Основные методы сбора данных в статистике включают различные подходы, которые позволяют получать информацию для последующего анализа. Эти методы могут быть качественными или количественными, в зависимости от цели исследования. Вот основные из них:
\newline\\
\textbf{Наблюдение}
\newline\\
\textbf{Описание:} Метод, при котором данные собираются через прямое или косвенное наблюдение за объектами, событиями или процессами без вмешательства в их естественное течение.
\newline\\
\textbf{Пример:} Наблюдение за поведением покупателей в магазине для определения их предпочтений.
\end{frame}
%------------------------------------------------

\begin{frame}
\frametitle{Основные методы сбора данных}
\textbf{Опрос}
\newline\\
\textbf{Описание:} Метод, при котором данные собираются с помощью анкет, интервью или тестов. Опросы могут быть структурированными (с заранее подготовленными вопросами) или неструктурированными (открытые вопросы).
\newline\\
\textbf{Пример:} Проведение анкетирования среди населения для выяснения уровня удовлетворенности услугами.
\end{frame}
%------------------------------------------------

\begin{frame}
\frametitle{Основные методы сбора данных}
\textbf{Эксперимент}
\newline\\
\textbf{Описание:} Метод, при котором исследователь активно вмешивается в процесс для изучения причинно-следственных связей. В ходе эксперимента изменяются определенные переменные, а затем наблюдаются эффекты этих изменений.
\newline\\
\textbf{Пример:} Тестирование новой формулы лекарства на группе пациентов, чтобы определить его эффективность.
\end{frame}
%------------------------------------------------

\begin{frame}
\frametitle{Основные методы сбора данных}
\textbf{Анализ документов и административных данных}
\newline\\
\textbf{Описание:} Сбор данных из уже существующих источников, таких как официальные документы, отчеты, административные базы данных, реестры и т. д.
\newline\\
\textbf{Пример:} Использование данных переписи населения или статистики преступности для анализа демографических тенденций.
\end{frame}
%------------------------------------------------

\begin{frame}
\frametitle{Основные методы сбора данных}
\textbf{Моделирование и симуляция}
\newline\\
\textbf{Описание:} Этот метод включает создание математических или компьютерных моделей, которые могут имитировать реальные процессы или явления. Данные собираются на основе выполнения этих моделей.
\newline\\
\textbf{Пример:} Симуляция климатических изменений для прогнозирования их влияния на окружающую среду.
\end{frame}
%------------------------------------------------

\begin{frame}
\frametitle{Основные методы сбора данных}
\textbf{Фокус-группы}
\newline\\
\textbf{Описание:} Метод, при котором группа людей собирается для обсуждения определенной темы, чтобы выявить мнения, убеждения и отношения участников.
\newline\\
\textbf{Пример:} Обсуждение новой рекламной кампании с группой потребителей для оценки их реакции на неё.
\end{frame}
%------------------------------------------------

\begin{frame}
\frametitle{Основные методы сбора данных}
\textbf{Панельное исследование}
\newline\\
\textbf{Описание:} Длительное исследование, при котором данные собираются от одной и той же группы людей (панели) в течение определенного времени. Это позволяет отслеживать изменения во времени.
\newline\\
\textbf{Пример:} Исследование потребительского поведения на протяжении нескольких лет для выявления долгосрочных тенденций.
\end{frame}
%------------------------------------------------

\begin{frame}
\frametitle{Основные методы сбора данных}
\textbf{Биг дата и автоматизированный сбор данных}
\newline\\
\textbf{Описание:} Современные методы, включающие использование больших объемов данных (big data), которые автоматически собираются с различных источников, таких как интернет, датчики, системы мониторинга и т. д.
\newline\\
\textbf{Пример:} Анализ данных с социальных сетей для предсказания трендов в обществе.
\newline\\
Каждый из этих методов имеет свои преимущества и ограничения и применяется в зависимости от целей и контекста исследования.
\end{frame}
%------------------------------------------------

\begin{frame}
\frametitle{Выборка и генеральная совокупность}
В статистике выборка и генеральная совокупность являются ключевыми понятиями, которые важны для проведения любого исследования и анализа данных.
\newline\\
\textbf{Генеральная совокупность}
\newline\\
Генеральная совокупность — это полное множество объектов, элементов или наблюдений, которые представляют интерес для исследования. Она включает в себя все возможные элементы, соответствующие определённым критериям, и может быть как конечной, так и бесконечной.
\newline\\
\textbf{Пример:} Если исследование направлено на изучение уровня доходов жителей страны, то генеральной совокупностью будут все жители этой страны.
\end{frame}
%------------------------------------------------

\begin{frame}
\frametitle{Выборка и генеральная совокупность}
\textbf{Выборка}
\newline\\
Выборка — это подмножество генеральной совокупности, которое отбирается для анализа. Выборка используется для того, чтобы сделать выводы о генеральной совокупности, поскольку в большинстве случаев исследовать всю генеральную совокупность невозможно или непрактично.
\newline\\
\textbf{Пример:} Если генеральная совокупность — это все жители страны, то выборка может включать, например, 1000 человек, случайным образом отобранных для проведения опроса.
\end{frame}
%------------------------------------------------

\begin{frame}
\frametitle{Выборка и генеральная совокупность}
\textbf{Важные аспекты выборки:}
\newline\\
\textbf{Репрезентативность:} Выборка должна быть репрезентативной, то есть отражать характеристики генеральной совокупности. Это важно для того, чтобы результаты анализа выборки могли быть обоснованно распространены на всю генеральную совокупность.
\newline\\
\textbf{Размер выборки:} Размер выборки должен быть достаточно большим, чтобы обеспечить точность оценок. Однако, он также должен быть сбалансирован с ресурсами, доступными для исследования.
\end{frame}
%------------------------------------------------

\begin{frame}
\frametitle{Выборка и генеральная совокупность}
\textbf{Методы отбора выборки:}
\newline\\
\textbf{Случайная выборка:} Каждый элемент генеральной совокупности имеет равную вероятность быть включенным в выборку. Это минимизирует предвзятость и увеличивает репрезентативность.
\newline\\
\textbf{Стратифицированная выборка:} Генеральная совокупность делится на подгруппы (страты), и из каждой подгруппы случайным образом отбирается выборка. Это помогает улучшить репрезентативность по важным характеристикам.
\end{frame}
%------------------------------------------------

\begin{frame}
\frametitle{Выборка и генеральная совокупность}
\textbf{Методы отбора выборки:}
\newline\\
\textbf{Кластерная выборка:} Генеральная совокупность делится на группы (кластеры), и случайным образом выбираются целые кластеры для исследования. Этот метод удобен, если невозможно собрать данные по всей генеральной совокупности.
\newline\\
\textbf{Систематическая выборка:} Элементы выборки отбираются через равные промежутки времени или пространства из упорядоченного списка. Например, каждый десятый человек из списка населения.
\end{frame}
%------------------------------------------------

\begin{frame}
\frametitle{Выборка и генеральная совокупность}
\textbf{Ошибки выборки:}
\newline\\
При работе с выборкой возможно возникновение ошибок:
\newline\\
\textbf{Ошибка выборки:} Разница между характеристиками выборки и генеральной совокупности. Она может возникнуть из-за случайных отклонений или систематических предвзятостей.
\newline\\
\textbf{Предвзятость выборки:} Систематическая ошибка, которая возникает, если выборка не является репрезентативной, например, если определённые группы населения недопредставлены.
\end{frame}
%------------------------------------------------

\begin{frame}
\frametitle{Выборка и генеральная совокупность}
\textbf{Применение в статистическом анализе:}
\newline\\
Исследователи используют данные выборки для оценки параметров генеральной совокупности, таких как среднее значение, дисперсия и т. д. Для этого применяются статистические методы, позволяющие сделать выводы о всей генеральной совокупности на основе анализа данных выборки.
\newline\\
\textbf{Пример:} Допустим, исследователь хочет узнать средний доход жителей города с населением 100 000 человек (генеральная совокупность). Проведение опроса всех жителей слишком затратно, поэтому он выбирает случайную выборку из 1000 человек. Средний доход в этой выборке может быть использован для оценки среднего дохода всего населения города, если выборка репрезентативна и достаточна по размеру.
\end{frame}
%------------------------------------------------

\begin{frame}
\frametitle{Выборка и генеральная совокупность}
Таким образом, выборка и генеральная совокупность — фундаментальные понятия в статистике, которые позволяют исследователям проводить анализ данных и делать выводы, применимые к большим группам объектов или людей.
\end{frame}
%------------------------------------------------

\begin{frame}
\frametitle{Частоты. Распределения}
В статистике частоты и распределения играют важную роль в описании и анализе данных. Эти понятия помогают структурировать данные и понять, как часто встречаются определённые значения или группы значений в выборке или генеральной совокупности.
\newline\\
\textbf{Частоты}
\newline\\
\textbf{Частота (или абсолютная частота)} — это количество раз, которое определённое значение или категория встречается в наборе данных.
\newline\\
\textbf{Пример:} В группе из 100 студентов 25 человек имеют оценку "5". В этом случае частота оценки "5" составляет 25.
\end{frame}
%------------------------------------------------

\begin{frame}
\frametitle{Частоты. Распределения}
\textbf{Относительная частота} — это доля или процент от общего числа наблюдений, соответствующих определённому значению.
\newline\\
\textbf{Пример:} Если из 100 студентов 25 человек получили оценку "5", то относительная частота этой оценки составляет 25/100 = 0,25 или 25\%.
\end{frame}
%------------------------------------------------

\begin{frame}
\frametitle{Частоты. Распределения}
\textbf{Распределения}
\newline\\
\textbf{Распределение} — это способ представления данных, показывающий, как часто встречаются различные значения или категории в выборке или генеральной совокупности. Распределение может быть представлено в виде таблиц, графиков или формул.
\newline\\
\textbf{Типы распределений:}
\newline\\
\textbf{Дискретное распределение частот:} Применяется для данных, которые принимают определённые, фиксированные значения (например, количество детей в семье).
\newline\\
\textbf{Пример:} В исследовании количества детей в семьях выборка может содержать семьи с 0, 1, 2, 3 и 4 детьми. Распределение покажет, сколько семей имеют каждый из этих вариантов.
\end{frame}
%------------------------------------------------

\begin{frame}
\frametitle{Частоты. Распределения}
\textbf{Типы распределений:}
\newline\\
\textbf{Непрерывное распределение частот:} Применяется для данных, которые могут принимать любые значения в определённом интервале (например, рост, вес).
\newline\\
\textbf{Пример:} Распределение роста среди студентов может показывать, что большинство студентов имеют рост в диапазоне 160-170 см, а менее распространённые значения находятся за пределами этого интервала.
\newline\\
\textbf{Распределение частот по категориям:} Применяется для категориальных данных (например, цвета, виды).
\newline\\
\textbf{Пример:} В исследовании предпочтений в цвете автомобилей распределение может показать, сколько людей предпочитают красный, синий, чёрный и другие цвета.
\end{frame}
%------------------------------------------------

\begin{frame}
\frametitle{Частоты. Распределения}
\textbf{Гистограмма и полигон частот}
\newline\\
\textbf{Гистограмма} — графическое представление распределения частот, где на горизонтальной оси откладываются интервалы значений, а на вертикальной — частоты (или относительные частоты). Высота столбцов показывает, сколько раз значения попадают в каждый интервал.
\newline\\
\textbf{Пример:} Гистограмма роста студентов может иметь столбцы, показывающие количество студентов в интервалах 150-160 см, 160-170 см и т. д.
\newline\\
\textbf{Полигон частот} — это линия, соединяющая точки, которые представляют частоты для каждого значения или интервала. Полигон частот обычно строится по центрам интервалов на гистограмме.
\end{frame}

\end{document}