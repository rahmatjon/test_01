\documentclass[aspectratio=169]{beamer}
%\documentclass{beamer}
\usepackage[utf8]{inputenc}
\usepackage[T2A]{fontenc}
\usepackage[russian]{babel}
\usepackage{cmap}
\usetheme{Boadilla}

\newcommand{\Ki}{\mathscr{K}}
\newcommand{\D}{\mathscr{D}}
\newcommand{\be}{\begin{equation}}
\newcommand{\ee}{\end{equation}}
\newcommand{\bes}{\begin{equation*}}
\newcommand{\ees}{\end{equation*}}

\makeatletter
\setbeamertemplate{footline}{%
  \leavevmode%
  \hbox{%
    \begin{beamercolorbox}[wd=.92\paperwidth,ht=2.25ex,dp=1ex,center]{title in head/foot}%
      \usebeamerfont{title in head/foot}\insertshorttitle
    \end{beamercolorbox}%
  }%
  \begin{beamercolorbox}[wd=.08\paperwidth,ht=2.25ex,dp=1ex,right]{date in head/foot}%
    \usebeamerfont{date in head/foot}%
    \usebeamertemplate{page number in head/foot}%
    \hspace*{2ex} 
  \end{beamercolorbox}
  \vskip0pt%
}
\makeatother

\title{\bf Урок 2. Описательные статистики}
\author{{\bf Хакимов Р.И. + ChatGPT}}
 \date[\today]{}

\begin{document}
\begin{frame}
\titlepage
\end{frame}

\begin{frame}
\frametitle{Статистические методы анализа данных}
Статистические методы анализа данных представляют собой набор техник и процедур, которые позволяют исследователям изучать, интерпретировать и делать выводы из данных. Эти методы варьируются от простых описательных статистик до сложных моделей и алгоритмов. Вот основные категории и примеры статистических методов анализа данных:
\newline\\
\end{frame}
%------------------------------------------------

\begin{frame}
\frametitle{Статистические методы анализа данных}
\textbf{Описательная статистика}
\newline\\
\textbf{Цель:} Обобщение и описание характеристик данных.
\newline\\
\textbf{Основные методы:}
\newline\\
\textit{Среднее значение (среднее арифметическое):} Среднее всех значений в наборе данных.
\newline\\
\textit{Медиана:} Центральное значение в упорядоченном наборе данных.
\newline\\
\textit{Мода:} Наиболее часто встречающееся значение.
\newline\\
\textit{Размах:} Разница между максимальным и минимальным значениями.
\end{frame}
%------------------------------------------------

\begin{frame}
\frametitle{Статистические методы анализа данных}
\textbf{Описательная статистика}
\newline\\
\textbf{Основные методы:}
\newline\\
\textit{Дисперсия и стандартное отклонение:} Измерение разброса данных относительно среднего значения.
\newline\\
\textit{Квартиль:} Деление данных на четыре равные части, позволяющее понять распределение данных.
\end{frame}
%------------------------------------------------

\begin{frame}
\frametitle{Статистические методы анализа данных}
\textbf{Инференциальная статистика / Статистический метод}
\newline\\
\textbf{Цель:} Сделать выводы о генеральной совокупности на основе данных выборки.
\newline\\
\textbf{Основные методы:}
\newline\\
\textbf{Доверительный интервал:} Интервал, в котором с определенной вероятностью находится истинное значение параметра генеральной совокупности.
\newline\\
\textbf{Тестирование гипотез:} Проверка предположений о параметрах генеральной совокупности на основе выборочных данных.

\quad\textit{$t$-тест:} Используется для сравнения средних значений двух групп.

\quad\textit{ANOVA (дисперсионный анализ):} Для сравнения средних значений более чем двух групп.

\quad\textit{$\chi^2$-тест (хи-квадрат тест):} Для проверки взаимосвязи между двумя категориальными переменными.
\end{frame}
%------------------------------------------------

\begin{frame}
\frametitle{Статистические методы анализа данных}
\textbf{Инференциальная статистика / Статистический метод}
\newline\\
\textbf{Основные методы:}
\newline\\
\textbf{Регрессионный анализ:} Метод, позволяющий оценить влияние одной или нескольких независимых переменных на зависимую переменную.

\quad\textit{Линейная регрессия:} Модель, предполагающая линейную связь между переменными.

\quad\textit{Логистическая регрессия:} Используется для моделирования вероятности бинарного исхода.

\end{frame}
%------------------------------------------------

\begin{frame}
\frametitle{Статистические методы анализа данных}
\textbf{Корреляционный анализ}
\newline\\
\textbf{Цель:} Оценка степени и направления связи между двумя переменными.
\newline\\
\textbf{Основные методы:}
\newline\\
\textbf{Коэффициент корреляции Пирсона:} Измеряет линейную зависимость между двумя количественными переменными.
\newline\\
\textbf{Коэффициент корреляции Спирмена:} Не параметрическая мера связи между ранжированными переменными.
\newline\\
\textbf{Ковариация:} Показатель, описывающий, как две переменные изменяются вместе.
\end{frame}
%------------------------------------------------

\begin{frame}
\frametitle{Статистические методы анализа данных}
\textbf{Многомерные методы анализа}
\newline\\
\textbf{Цель:} Изучение зависимости и связей между несколькими переменными одновременно.
\newline\\
\textbf{Основные методы:}
\newline\\
\textbf{Множественная регрессия:} Расширение линейной регрессии для случая, когда есть несколько независимых переменных.
\newline\\
\textbf{Кластерный анализ:} Метод группировки объектов или наблюдений в группы (кластеры) на основе сходства между ними.
\end{frame}
%------------------------------------------------

\begin{frame}
\frametitle{Статистические методы анализа данных}
\textbf{Временные ряды}
\newline\\
\textbf{Цель:} Анализ данных, упорядоченных во времени.
\newline\\
\textbf{Основные методы:}
\newline\\
\textbf{Автокорреляция:} Измерение зависимости данных от своих прошлых значений.
\newline\\
\textbf{Модели ARIMA (авторегрессия, интегрированная модель скользящего среднего):} Для прогнозирования временных рядов.
\newline\\
\textbf{Экспоненциальное сглаживание:} Метод для сглаживания временных рядов и прогноза будущих значений.
\end{frame}
%------------------------------------------------

\begin{frame}
\frametitle{Статистические методы анализа данных}
\textbf{Байесовские методы}
\newline\\
\textbf{Цель:} Обновление вероятности гипотезы на основе новых данных.
\newline\\
\textbf{Основные методы:}
\newline\\
\textbf{Байесовская регрессия:} Обновление распределения параметров модели по мере поступления новых данных.
\newline\\
\textbf{Байесовская сеть:} Графическая модель, которая представляет зависимость между переменными.
\end{frame}
%------------------------------------------------

\begin{frame}
\frametitle{Статистические методы анализа данных}
\textbf{Методы машинного обучения и статистическое обучение}
\newline\\
\textbf{Цель:} Создание моделей, которые могут предсказывать или классифицировать данные на основе обучения на существующих данных.
\newline\\
\textbf{Основные методы:}
\newline\\
\textbf{Методы классификации:} Логистическая регрессия, деревья решений, случайные леса, нейронные сети.
\newline\\
\textbf{Методы регрессии:} Линейная и нелинейная регрессия, опорные векторы.
\newline\\
\textbf{Методы кластеризации:} k-средних, иерархическая кластеризация.
\end{frame}
%------------------------------------------------

\begin{frame}
\frametitle{Статистические методы анализа данных}
Эти методы и техники позволяют исследователям и аналитикам собирать, анализировать и интерпретировать данные, делая обоснованные выводы и принимая решения. Выбор конкретного метода зависит от целей исследования, природы данных и гипотез, которые нужно проверить.
\end{frame}
%------------------------------------------------

\begin{frame}
\frametitle{Меры центральной тенденции: среднее значение, медиана, мода}
Меры центральной тенденции — это статистические показатели, которые описывают центр распределения данных. Они помогают понять, где сосредоточены значения в наборе данных. Вот основные меры центральной тенденции:
\end{frame}
%------------------------------------------------

\begin{frame}
\frametitle{Меры центральной тенденции: среднее значение, медиана, мода}
\textbf{Среднее значение (среднее арифметическое)}

\textbf{Описание:} Среднее значение рассчитывается как сумма всех значений в наборе данных, деленная на количество этих значений. Это наиболее распространенная мера центральной тенденции.

\textbf{Формула:}
  \[
  \bar{X} = \frac{\sum\limits_{i=1}^{n} X_i}{n}
  \]
где \(\bar{X}\) — среднее значение, \(X_i\) — отдельные значения, \(n\) — количество значений.

\textbf{Пример:} Если у вас есть набор данных: 4, 7, 8, 10, то среднее значение будет:
  \[
  \bar{X} = \frac{4 + 7 + 8 + 10}{4} = 7.25
  \]

\textbf{Особенности:} Среднее значение чувствительно к экстремальным значениям (выбросам), что может искажать представление о центральной тенденции в случае наличия аномально больших или маленьких значений.
\end{frame}
%------------------------------------------------

\begin{frame}
\frametitle{Меры центральной тенденции: среднее значение, медиана, мода}
\textbf{Медиана}

\textbf{Описание:} Медиана — это значение, которое делит упорядоченный набор данных пополам. Половина значений в наборе меньше медианы, а другая половина больше.

\textbf{Формула:}

\quad- Для нечетного числа значений медиана — это срединное значение.

\quad- Для четного числа значений медиана — это среднее арифметическое двух срединных значений.

\textbf{Пример:}

\quad- Если есть набор данных: 3, 7, 8, 12, 14, то медиана будет 8 (третье значение в упорядоченном наборе).

\quad- Если есть набор данных: 3, 7, 8, 12, 14, 20, то медиана будет \(\frac{8 + 12}{2} = 10\) (среднее арифметическое двух срединных значений).

\textbf{Особенности:} Медиана не чувствительна к выбросам и экстраординарным значениям, поэтому она лучше отражает центральную тенденцию в случаях, когда данные имеют сильные отклонения.
\end{frame}
%------------------------------------------------

\begin{frame}
\frametitle{Меры центральной тенденции: среднее значение, медиана, мода}
\textbf{Мода}

\textbf{Описание:} Мода — это значение, которое встречается в наборе данных наиболее часто. В отличие от среднего значения и медианы, мода может быть не уникальной: набор данных может иметь несколько мод (многомодальный) или не иметь мод вообще.

\textbf{Пример:}

\quad- Если есть набор данных: 1, 2, 2, 3, 4, то мода будет 2 (значение, которое встречается чаще всего).

\quad- Если есть набор данных: 1, 2, 3, 4, 5, то в этом случае мода отсутствует, так как все значения встречаются одинаково часто.

\textbf{Особенности:} Мода полезна для категориальных данных, где вычисление среднего значения или медианы может быть неуместным. Также мода может давать представление о наиболее частом событии или явлении в данных.
\end{frame}
%------------------------------------------------

\begin{frame}
\frametitle{Меры центральной тенденции: среднее значение, медиана, мода}
\textbf{Сравнение методов}
\newline\\
\textbf{Среднее значение} часто используется, когда данные распределены нормально и нет значительных выбросов. Оно предоставляет хорошее представление о "центре" данных.
\newline\\
\textbf{Медиана} предпочтительна, когда данные имеют выбросы или распределение несимметрично, так как она не искажается экстремальными значениями.
\newline\\
\textbf{Мода} полезна для категориальных данных и для понимания наиболее частых значений в данных, но может быть менее информативной для количественных данных, особенно если значения распределены равномерно.
\newline\\
В разных ситуациях различные меры центральной тенденции могут предоставлять более полезную информацию, поэтому важно учитывать особенности данных при выборе подходящего метода.
\end{frame}
%------------------------------------------------

\begin{frame}
\frametitle{Меры изменчивости: диапазон, дисперсия, стандартное отклонение}
Меры изменчивости помогают оценить, насколько данные в наборе варьируются или отклоняются от центра распределения. Они дают представление о том, насколько значения данных разбросаны вокруг центральной тенденции (например, среднего значения). Вот основные меры изменчивости:
\end{frame}
%------------------------------------------------

\begin{frame}
\frametitle{Меры изменчивости: диапазон, дисперсия, стандартное отклонение}
\textbf{Диапазон, размах}

\textbf{Описание:} Диапазон — это разница между максимальным и минимальным значениями в наборе данных. Это самая простая мера изменчивости.

\textbf{Формула:}
  \[
  \text{Диапазон} = X_{\text{max}} - X_{\text{min}}
  \]
где \(X_{\text{max}}\) — максимальное значение, а \(X_{\text{min}}\) — минимальное значение.

\textbf{Пример:} Для набора данных 5, 8, 12, 15 диапазон будет:
  \[
  15 - 5 = 10
  \]

\textbf{Особенности:} Диапазон прост в расчетах и интерпретации, но он чувствителен к выбросам, так как зависит только от крайних значений.
\end{frame}
%------------------------------------------------

\begin{frame}
\frametitle{Меры изменчивости: диапазон, дисперсия, стандартное отклонение}
\textbf{Дисперсия}

\textbf{Описание:} Дисперсия — это среднее значение квадратов отклонений наблюдений от их среднего значения. Она показывает, насколько сильно значения данных отклоняются от среднего.

\textbf{Формула:}
  \[
  \sigma^2 = \frac{1}{N} \sum_{i=1}^{N} (X_i - \bar{X})^2
  \]
где \(\sigma^2\) — дисперсия, \(X_i\) — отдельные значения, \(\bar{X}\) — среднее значение, \(N\) — количество значений.


Для выборки формула дисперсии немного изменяется:
  \[
  s^2 = \frac{1}{n-1} \sum_{i=1}^{n} (X_i - \bar{X})^2
  \]
где \(s^2\) — выборочная дисперсия, \(n\) — размер выборки.
\end{frame}
%------------------------------------------------

\begin{frame}
\frametitle{Меры изменчивости: диапазон, дисперсия, стандартное отклонение}
\textbf{Дисперсия}

\textbf{Пример:} Для набора данных 4, 8, 6, 5, 9:
  \[
  \bar{X} = \frac{4 + 8 + 6 + 5 + 9}{5} = 6.4
  \]

дисперсия равна:
  \[
  \sigma^2 = \frac{(4-6.4)^2 + (8-6.4)^2 + (6-6.4)^2 + (5-6.4)^2 + (9-6.4)^2}{5}
  \]
  \[
  = \frac{5.76 + 2.56 + 0.16 + 1.96 + 6.76}{5} = 3.44
  \]

\textbf{Особенности:} Дисперсия выражается в квадрате единиц измерения исходных данных, что может затруднять интерпретацию.
\end{frame}
%------------------------------------------------

\begin{frame}
\frametitle{Меры изменчивости: диапазон, дисперсия, стандартное отклонение}
\textbf{Стандартное отклонение}

\textbf{Описание:} Стандартное отклонение — это квадратный корень из дисперсии. Оно измеряет среднее отклонение значений от среднего значения и выражается в тех же единицах измерения, что и исходные данные.

\textbf{Формула:}
  \[
  \sigma = \sqrt{ \frac{1}{N} \sum_{i=1}^{N} (X_i - \bar{X})^2}
  \]
где \(\sigma\) — стандартное отклонение, \(X_i\) — отдельные значения, \(\bar{X}\) — среднее значение, \(N\) — количество значений.


Для выборки:
  \[
   \sigma = \sqrt{ \frac{1}{n-1} \sum_{i=1}^{n} (X_i - \bar{X})^2}
  \]
где \(\sigma\) — стандартное отклонение, \(n\) — размер выборки.
\end{frame}
%------------------------------------------------

\begin{frame}
\frametitle{Меры изменчивости: диапазон, дисперсия, стандартное отклонение}
\textbf{Стандартное отклонение}

\textbf{Пример:} Для вышеупомянутого набора данных, если дисперсия равна 3.44, то стандартное отклонение будет:
  \[
  \sigma = \sqrt{3.44} \approx 1.85
  \]

\textbf{Особенности:} Стандартное отклонение более интуитивно понятно, чем дисперсия, так как оно выражается в тех же единицах измерения, что и исходные данные.
\end{frame}
%------------------------------------------------

\begin{frame}
\frametitle{Меры изменчивости: диапазон, дисперсия, стандартное отклонение}
\textbf{Сравнение мер изменчивости}

\textbf{Диапазон} предоставляет общую информацию о разбросе данных, но не учитывает распределение внутри интервала и чувствителен к выбросам.

\textbf{Дисперсия} дает более полное представление о том, насколько данные варьируются вокруг среднего значения, но ее интерпретация может быть затруднена из-за квадратов единиц измерения.

\textbf{Стандартное отклонение} является более удобной мерой для интерпретации, так как оно находится в тех же единицах, что и данные, и дает представление о средней степени отклонения значений от среднего.
\newline\\
Эти меры помогают исследователям и аналитикам понять, насколько данные разбросаны и как это может влиять на интерпретацию результатов и выводы из анализа.
\end{frame}
%------------------------------------------------

\end{document}