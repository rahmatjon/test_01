\documentclass[aspectratio=169]{beamer}
%\documentclass{beamer}
\usepackage[utf8]{inputenc}
\usepackage[T2A]{fontenc}
\usepackage[russian]{babel}
\usepackage{cmap}
\usetheme{Boadilla}

\newcommand{\Ki}{\mathscr{K}}
\newcommand{\D}{\mathscr{D}}
\newcommand{\be}{\begin{equation}}
\newcommand{\ee}{\end{equation}}
\newcommand{\bes}{\begin{equation*}}
\newcommand{\ees}{\end{equation*}}

\makeatletter
\setbeamertemplate{footline}{%
  \leavevmode%
  \hbox{%
    \begin{beamercolorbox}[wd=.92\paperwidth,ht=2.25ex,dp=1ex,center]{title in head/foot}%
      \usebeamerfont{title in head/foot}\insertshorttitle
    \end{beamercolorbox}%
  }%
  \begin{beamercolorbox}[wd=.08\paperwidth,ht=2.25ex,dp=1ex,right]{date in head/foot}%
    \usebeamerfont{date in head/foot}%
    \usebeamertemplate{page number in head/foot}%
    \hspace*{2ex} 
  \end{beamercolorbox}
  \vskip0pt%
}
\makeatother

\title{\bf Урок 3. Теория вероятностей (часть 1)}
\author{{\bf Хакимов Р.И. + ChatGPT}}
 \date[\today]{}

\begin{document}
\begin{frame}
\titlepage
\end{frame}

\begin{frame}
\frametitle{Случайные события, величины, свойства и операции}
Теория вероятностей изучает случайные события и их вероятности. Основные понятия включают случайные события, случайные величины, а также свойства и операции с ними. Далее обзор этих понятий.
\end{frame}
%------------------------------------------------

\begin{frame}
\frametitle{Случайные события, величины, свойства и операции}
\textbf{Случайные события}
\newline\\
\textbf{Описание:} Случайное событие — это результат, который может произойти в рамках некоторого случайного эксперимента. Событие может быть элементарным (то есть состоящим из одного исхода) или составным (состоящим из нескольких исходов).
\newline\\
\textbf{Пример:} В подбрасывании монеты "орел" или "решка" — это случайные события. Если рассматривается подбрасывание двух монет, событие "обе монеты показывают орел" является составным событием.
\end{frame}
%------------------------------------------------

\begin{frame}
\frametitle{Случайные события, величины, свойства и операции}
\textbf{Типы событий}
\newline\\
\textbf{Элементарное событие:} Событие, которое не может быть разложено на более простые события.

\textbf{Пример:} Падение одной монеты "орел" или "решка".
\newline\\
\textbf{Совместные события:} События, которые могут происходить одновременно.

\textbf{Пример:} В подбрасывании двух монет "первая монета показывает орел" и "вторая монета показывает решку".
\newline\\
\textbf{Противоположные события:} События, которые не могут произойти одновременно.

\textbf{Пример:} "Монета показывает орел" и "монета показывает решку".
\newline\\
\textbf{Независимые события:} События, где наступление одного не влияет на вероятность наступления другого.

\textbf{Пример:} Подбрасывание двух монет.
\end{frame}
%------------------------------------------------

\begin{frame}
\frametitle{Случайные события, величины, свойства и операции}
\textbf{Случайные величины}
\newline\\
\textbf{Описание:} Случайная величина — это функция, которая присваивает числовые значения каждому возможному исходу случайного эксперимента.
\newline\\
\textbf{Типы случайных величин:}
\newline\\
\textbf{Дискретные случайные величины:} Могут принимать конечное или счетное число значений.
\newline
\textbf{Пример:} Количество орлов при подбрасывании двух монет (0, 1, 2).
\newline\\
\textbf{Непрерывные случайные величины:} Могут принимать любые значения в некотором интервале.
\newline
\textbf{Пример:} Рост человека, который может варьироваться от 150 до 200 см.
\end{frame}
%------------------------------------------------

\begin{frame}
\frametitle{Вероятность: определения}
В теории вероятностей вероятность — это числовая мера того, насколько вероятно, что определённое событие произойдет. Существуют различные подходы к определению вероятности, которые можно классифицировать на несколько основных типов:
\end{frame}
%------------------------------------------------

\begin{frame}
\frametitle{Вероятность: определения}
\textbf{Классическое определение вероятности}
\newline
\textbf{Описание:} Вероятность события определяется как отношение числа благоприятных исходов к общему числу возможных исходов, при условии что все исходы равновероятны.
\newline
\textbf{Формула:}
  \[
  P(A) = \frac{\text{Количество благоприятных исходов}}{\text{Общее количество возможных исходов}}
  \]
\newline
\textbf{Пример:} При броске симметричной шестигранной кости вероятность выпадения числа 4:
  \[
  P(\text{выпадение 4}) = \frac{1}{6}
  \]
так как есть 1 благоприятный исход (выпадение 4) и 6 возможных исходов.
\end{frame}
%------------------------------------------------

\begin{frame}
\frametitle{Вероятность: определения}
\textbf{Частотное определение вероятности}
\newline
\textbf{Описание:} Вероятность события определяется как отношение числа успешных попыток (или наблюдений), когда событие произошло, к общему числу попыток.
\newline
\textbf{Формула:}
  \[
  P(A) = \frac{\text{Число успешных попыток}}{\text{Общее число попыток}}
  \]
\newline
\textbf{Пример:} Если при подбрасывании монеты 100 раз орел выпадает 45 раз, то частотная вероятность выпадения орла:
  \[
  P(\text{орел}) = \frac{45}{100} = 0.45
  \]
\end{frame}
%------------------------------------------------

\begin{frame}
\frametitle{Вероятность: определения}
\textbf{Аксиоматическое определение вероятности}
\newline
\textbf{Описание:} Основывается на аксиомах, предложенных Андреем Колмогоровым. Это наиболее общепринятый и формализованный способ определения вероятности. Вероятность — это функция, которая удовлетворяет определенным аксиомам.
\newline
\textbf{Аксиомы Колмогорова:}
\newline
\textbf{1. Ненегативность:} \( P(A) \geq 0 \) для любого события \( A \).
\newline
\textbf{2. Нормированность:} \( P(\Omega) = 1 \), где \( \Omega \) — пространство всех возможных исходов (или универсум).
\newline
\textbf{3. Счётная аддитивность ($\sigma$-аддитивность):} Если события \( A_1, A_2, A_3, \dots \) попарно непересекающиеся (\( A_i \cap A_j = \emptyset \) при \( i \neq j \)), то:  
   \[
   P\left(\bigcup_{i=1}^{\infty} A_i\right) = \sum_{i=1}^{\infty} P(A_i)
   \]
\textbf{Пример:} Пусть есть три дискретных непересекающихся события \( A \), \( B \) и \( C \), которые составляют пространство элементарных исходов. Тогда вероятность их объединения будет:
  \[
  P(A \cup B \cup C) = P(A) + P(B) + P(C)
  \]
\end{frame}
%------------------------------------------------

\begin{frame}
\frametitle{Вероятность: определения}
\textbf{Субъективное определение вероятности}
\newline
\textbf{Описание:} Вероятность события рассматривается как мера уверенности или субъективного мнения о том, что событие произойдет, и может варьироваться в зависимости от личного опыта или знаний.
\newline
\textbf{Пример:} Если кто-то считает, что шансы на победу своей любимой команды в следующем матче равны 70\%, то это субъективная вероятность, основанная на личном мнении и интуиции.
\end{frame}
%------------------------------------------------

\begin{frame}
\frametitle{Случайные события, величины, свойства и операции}
\textbf{Функции случайной величины}
\newline\\
\textbf{Функция распределения (CDF, Cumulative Distribution Function):} Функция, которая показывает вероятность того, что случайная величина примет значение меньше или равно заданному.
\newline
\textbf{Формула:} \( F(x) = P(X \leq x) \), где \( F(x) \) — функция распределения, \( X \) — случайная величина.
\newline
\textbf{Функция вероятностей (для дискретных случайных величин) (PMF, Probability Mass Function):} Определяет вероятность того, что случайная величина примет конкретное значение.
\newline
\textbf{Формула:} \( P(X = x) \), где \( P(X = x) \) — вероятность того, что \( X \) равна \( x \).
\end{frame}
%------------------------------------------------

\begin{frame}
\frametitle{Свойства случайных величин}
\textbf{Ожидаемое значение (математическое ожидание)}
\newline\\
\textbf{Описание:} Среднее значение случайной величины, которое ожидается в долгосрочной перспективе.
\newline
\textbf{Формула (для дискретных случайных величин):}
  \[
  E(X) = \sum_{i} x_i \cdot P(X = x_i)
  \]
\newline
\textbf{Формула (для непрерывных случайных величин):}
  \[
  E(X) = \int_{-\infty}^{\infty} x \cdot f(x) \, dx
  \]
где \( f(x) \) — функция плотности вероятности.
\end{frame}
%------------------------------------------------

\begin{frame}
\frametitle{Свойства случайных величин}
\textbf{Дисперсия}
\newline\\
\textbf{Описание:} Среднее значение квадратов отклонений случайной величины от её математического ожидания.
\newline
\textbf{Формула:}
  \[
  \text{Var}(X) = E[(X - E(X))^2]
  \]
или
  \[
  \text{Var}(X) = E(X^2) - (E(X))^2
  \]
\end{frame}
%------------------------------------------------

\begin{frame}
\frametitle{Свойства случайных величин}
\textbf{Стандартное отклонение}
\newline\\
\textbf{Описание:} Квадратный корень из дисперсии. Показывает среднее отклонение значений случайной величины от её математического ожидания.
\newline
\textbf{Формула:}
  \[
  \sigma = \sqrt{\text{Var}(X)}
  \]
\end{frame}
%------------------------------------------------

\begin{frame}
\frametitle{Операции с событиями}
\textbf{Объединение событий}
\newline\\
\textbf{Описание:} Событие, что происходит одно из двух (или более) событий.
\newline
\textbf{Формула:}
  \[
  P(A \cup B) = P(A) + P(B) - P(A \cap B)
  \]
где \( P(A \cup B) \) — вероятность объединения событий \( A \) и \( B \), \( P(A \cap B) \) — вероятность их пересечения.
\newline\\
\textbf{Пересечение событий}
\newline\\
\textbf{Описание:} Событие, что происходят оба (или несколько) событий одновременно.
\newline
\textbf{Формула:}
  \[
  P(A \cap B) = P(A) \cdot P(B) \text{ (если события независимы)}
  \]
\end{frame}
%------------------------------------------------

\begin{frame}
\frametitle{Операции с событиями}
\textbf{Дополнение события}
\newline\\
\textbf{Описание:} Событие, что исход не принадлежит заданному событию.
\newline
\textbf{Формула:}
  \[
  P(A^c) = 1 - P(A)
  \]
где \( A^c \) — дополнение события \( A \).
\newline\\
\textbf{Условная вероятность}
\newline\\
\textbf{Описание:} Вероятность наступления события \( A \) при условии, что событие \( B \) уже произошло.
\newline
\textbf{Формула:}
  \[
  P(A | B) = \frac{P(A \cap B)}{P(B)}
  \]
\end{frame}
%------------------------------------------------

\begin{frame}
\frametitle{Законы вероятности}
\textbf{Закон сложения}
\newline\\
\textbf{Описание:} Вероятность того, что произойдет хотя бы одно из нескольких событий.
\newline
\textbf{Формула:}
  \[
  P(A \cup B) = P(A) + P(B) - P(A \cap B)
  \]
\newline\\
\textbf{Закон умножения}
\newline\\
\textbf{Описание:} Вероятность того, что произойдут оба события.
\newline
\textbf{Формула (для независимых событий):}
  \[
  P(A \cap B) = P(A) \cdot P(B)
  \]
\end{frame}

\end{document}