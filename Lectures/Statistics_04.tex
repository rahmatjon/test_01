\documentclass[aspectratio=169]{beamer}
%\documentclass{beamer}
\usepackage[utf8]{inputenc}
\usepackage[T2A]{fontenc}
\usepackage[russian]{babel}
\usepackage{cmap}
\usetheme{Boadilla}

\newcommand{\Ki}{\mathscr{K}}
\newcommand{\D}{\mathscr{D}}
\newcommand{\be}{\begin{equation}}
\newcommand{\ee}{\end{equation}}
\newcommand{\bes}{\begin{equation*}}
\newcommand{\ees}{\end{equation*}}

\makeatletter
\setbeamertemplate{footline}{%
  \leavevmode%
  \hbox{%
    \begin{beamercolorbox}[wd=.92\paperwidth,ht=2.25ex,dp=1ex,center]{title in head/foot}%
      \usebeamerfont{title in head/foot}\insertshorttitle
    \end{beamercolorbox}%
  }%
  \begin{beamercolorbox}[wd=.08\paperwidth,ht=2.25ex,dp=1ex,right]{date in head/foot}%
    \usebeamerfont{date in head/foot}%
    \usebeamertemplate{page number in head/foot}%
    \hspace*{2ex} 
  \end{beamercolorbox}
  \vskip0pt%
}
\makeatother

\title{\bf Урок 4. Теория вероятностей (часть2)}
\author{{\bf Хакимов Р.И. + ChatGPT}}
 \date[\today]{}

\begin{document}
\begin{frame}
\titlepage
\end{frame}
%------------------------------------------------

\begin{frame}
\frametitle{Комбинаторика: перестановки, сочетания, размещения}
Комбинаторика в теории вероятностей и статистике изучает способы, которыми можно выбирать и упорядочивать элементы из множества. Основные понятия включают перестановки, сочетания и размещения. Вот основные определения и формулы для этих понятий:
\end{frame}
%------------------------------------------------

\begin{frame}
\frametitle{Комбинаторика: перестановки, сочетания, размещения}
\textbf{Перестановки}
\newline
\textbf{Описание:} Перестановка — это упорядоченный набор элементов. Если у вас есть \( n \) уникальных элементов, то количество возможных перестановок этих элементов — это количество способов, которыми можно упорядочить все элементы.
\newline
\textbf{Формула для количества перестановок \( n \) элементов:}
  \[
  P(n) = P_n = n!
  \]
где \( n! \) (факториал \( n \)) — это произведение всех натуральных чисел от 1 до \( n \):
  \[
  n! = n \times (n-1) \times (n-2) \times \ldots \times 1
  \]
\newline
\textbf{Пример:} Для 3 элементов \( A, B, C \), возможные перестановки будут: ABC, ACB, BAC, BCA, CAB, CBA. Всего \( 3! = 6 \) перестановок.
\end{frame}
%------------------------------------------------

\begin{frame}
\frametitle{Комбинаторика: перестановки, сочетания, размещения}
\textbf{Сочетания}
\newline
\textbf{Описание:} Сочетание — это выбор подмножества элементов из множества без учета порядка. Если нужно выбрать \( k \) элементов из \( n \) доступных, не учитывая порядок, то это сочетание.
\newline
\textbf{Формула для сочетаний \( k \) элементов из \( n \) (без повторений):}
  \[
  \binom{n}{k} = C(n, k) = C_n^k = \frac{n!}{k! \cdot (n - k)!}
  \]
\textbf{Пример:} Если у вас есть 5 книг, и вы хотите выбрать 2, то количество способов это сделать:
  \[
  C_5^2 = \frac{5!}{2! \cdot (5 - 2)!} = \frac{120}{2 \cdot 6} = 10
  \]
Таким образом, есть 10 способов выбрать 2 книги из 5.
\end{frame}
%------------------------------------------------

\begin{frame}
\frametitle{Комбинаторика: перестановки, сочетания, размещения}
\textbf{Размещения}
\newline
\textbf{Описание:} Размещение — это выбор и упорядочивание подмножества элементов из множества. В отличие от сочетаний, здесь важен порядок выбранных элементов.
\newline
\textbf{Формула для размещений \( k \) элементов из \( n \):}
  \[
  A(n, k) = A_n^k = \frac{n!}{(n - k)!}
  \]
\textbf{Пример:} Если у вас есть 4 книги, и вы хотите выбрать 2 и упорядочить их, количество способов это сделать:
  \[
  A_4^2 = \frac{4!}{(4 - 2)!} = \frac{24}{2} = 12
  \]
Таким образом, есть 12 способов выбрать 2 книги из 4 и упорядочить их.
\end{frame}
%------------------------------------------------

\begin{frame}
\frametitle{Комбинаторика: перестановки, сочетания, размещения}
\textbf{Примеры применения}
\newline\\
\textbf{Перестановки:} При организации конкурсов, где важно определить, кто займет какое место, используется перестановка. Например, сколько способов упорядочить 5 призеров в конкурсе.
\newline\\
\textbf{Сочетания:} Выбор комбинаций команд из игроков, где порядок не имеет значения. Например, сколько способов выбрать 3 игрока из 10 для участия в команде.
\newline\\
\textbf{Размещения:} Распределение задач между членами команды, где важен порядок распределения. Например, сколько способов назначить 3 из 10 сотрудников на конкретные задачи, если порядок назначения имеет значение.
\newline\\
Эти комбинаторные методы являются основой для решения задач, связанных с подсчетом способов выполнения различных комбинаций и упорядочиванием элементов, что широко используется в теории вероятностей, статистике и других областях науки.
\end{frame}
%------------------------------------------------

\begin{frame}
\frametitle{Условная вероятность}
\textbf{Условная вероятность} — это вероятность наступления события \( A \), при условии, что событие \( B \) уже произошло. Условная вероятность показывает, как изменяется вероятность события \( A \) в свете дополнительной информации о событии \( B \).
\newline\\
\textbf{Формальное определение}
\newline
Условная вероятность события \( A \), при условии, что событие \( B \) уже произошло, обозначается как \( P(A | B) \). Она определяется следующим образом:

\[
P(A | B) = \frac{P(A \cap B)}{P(B)}
\]
где:\\
- \( P(A | B) \) — условная вероятность события \( A \) при условии, что произошло событие \( B \).\\
- \( P(A \cap B) \) — вероятность одновременного наступления событий \( A \) и \( B \).\\
- \( P(B) \) — вероятность наступления события \( B \), при условии что \( P(B) > 0 \).
\end{frame}
%------------------------------------------------

\begin{frame}
\frametitle{Условная вероятность}
\textbf{Свойства условной вероятности} 
\newline
\textbf{1. Неотрицательность:}
   \[
   P(A | B) \geq 0
   \]
\textbf{2. Нормированность:}
   \[
   P(B) = \sum_{i} P(A_i | B) \quad \text{для всех возможных } A_i
   \]
где \( A_i \) — всевозможные события, которые покрывают пространство.
\newline\\
\textbf{3. Условная вероятность не может быть больше 1:}
   \[
   0 \leq P(A | B) \leq 1
   \]
\textbf{4. Если \( A \subseteq B \):}
   \[
   P(A | B) = \frac{P(A)}{P(B)}
   \]
\end{frame}
%------------------------------------------------

\begin{frame}
\frametitle{Условная вероятность}
\textbf{Пример с игральной костью} 
\newline\\
Пусть есть игральная кость. Мы хотим узнать вероятность того, что на верхней грани выпадет число 4 (событие \( A \)), если мы знаем, что число на грани четное (событие \( B \)).
\newline\\
- Вероятность того, что выпадет четное число (2, 4, или 6), равна \( P(B) = \frac{3}{6} = \frac{1}{2} \).
\newline\\
- Вероятность того, что выпадет число 4 и оно четное, равна \( P(A \cap B) = \frac{1}{6} \) (так как только 4 из четных чисел совпадает с событием \( A \)).
\newline\\
Таким образом, условная вероятность выпадения 4, при условии, что число четное:
   \[
   P(A | B) = \frac{P(A \cap B)}{P(B)} = \frac{\frac{1}{6}}{\frac{1}{2}} = \frac{1}{3}
   \]
\end{frame}
%------------------------------------------------

\begin{frame}
\frametitle{Условная вероятность}
\textbf{Пример с картами} 
\newline\\
Пусть у нас есть колода из 52 карт, и мы извлекаем одну карту. Мы хотим найти вероятность того, что карта — это туз (событие \( A \)), если мы знаем, что карта — это червовая (событие \( B \)).
\newline\\
- Вероятность того, что карта червовая, \( P(B) = \frac{13}{52} = \frac{1}{4} \).
\newline\\
- Вероятность того, что карта — это туз и она червовая, \( P(A \cap B) = \frac{1}{52} \).
\newline\\
Условная вероятность того, что карта — это туз, при условии, что она червовая:
   \[
   P(A | B) = \frac{P(A \cap B)}{P(B)} = \frac{\frac{1}{52}}{\frac{1}{4}} = \frac{1}{13}
   \]
\end{frame}
%------------------------------------------------

\begin{frame}
\frametitle{Законы, связанные с условной вероятностью}
\textbf{Формула полной вероятности:} Если события \( B_1, B_2, \ldots, B_n \) образуют полную группу (то есть они покрывают всё пространство), то для любого события \( A \) справедлива формула:
   \[
   P(A) = \sum_{i=1}^{n} P(A | B_i) \cdot P(B_i)
   \]
\textbf{Формула Байеса:} Позволяет вычислять условные вероятности, зная вероятность обратного события:
   \[
   P(B_i | A) = \frac{P(A | B_i) \cdot P(B_i)}{\sum_{j=1}^{n} P(A | B_j) \cdot P(B_j)}
   \]
где \( B_i \) — одно из событий, образующих полную группу.
\newline\\
Условная вероятность позволяет анализировать вероятность события в зависимости от дополнительных условий или информации, что особенно полезно в различных областях науки и практики, таких как медицинская диагностика, статистика, инженерия и экономика.
\end{frame}

\end{document}