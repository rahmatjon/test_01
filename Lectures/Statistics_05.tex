\documentclass[aspectratio=169]{beamer}
%\documentclass{beamer}
\usepackage[utf8]{inputenc}
\usepackage[T2A]{fontenc}
\usepackage[russian]{babel}
\usepackage{cmap}
\usetheme{Boadilla}

\newcommand{\Ki}{\mathscr{K}}
\newcommand{\D}{\mathscr{D}}
\newcommand{\be}{\begin{equation}}
\newcommand{\ee}{\end{equation}}
\newcommand{\bes}{\begin{equation*}}
\newcommand{\ees}{\end{equation*}}

\makeatletter
\setbeamertemplate{footline}{%
  \leavevmode%
  \hbox{%
    \begin{beamercolorbox}[wd=.92\paperwidth,ht=2.25ex,dp=1ex,center]{title in head/foot}%
      \usebeamerfont{title in head/foot}\insertshorttitle
    \end{beamercolorbox}%
  }%
  \begin{beamercolorbox}[wd=.08\paperwidth,ht=2.25ex,dp=1ex,right]{date in head/foot}%
    \usebeamerfont{date in head/foot}%
    \usebeamertemplate{page number in head/foot}%
    \hspace*{2ex} 
  \end{beamercolorbox}
  \vskip0pt%
}
\makeatother

\title{\bf Урок 5. Распределения случайных величин (часть 1)}
\author{{\bf Хакимов Р.И. + ChatGPT}}
 \date[\today]{}

\begin{document}
\begin{frame}
\titlepage
\end{frame}
%------------------------------------------------

\begin{frame}
\frametitle{Введение в распределения случайных величин}
{\bf Распределения случайных величин} описывают, как вероятности распределяются среди возможных значений случайной величины. Введение в распределения случайных величин включает понимание их основных типов, функций и свойств. Далее преведен обзор ключевых понятий и распределений.
\end{frame}
%------------------------------------------------

\begin{frame}
\frametitle{Основные понятия}
{\bf 1. Случайная величина:} Функция, которая каждому элементарному исходу случайного эксперимента сопоставляет числовое значение.
\newline\\
{\bf 2. Функция распределения вероятностей:}\\
{\bf - Для дискретных случайных величин:} Функция вероятностей \( P(X = x) \), где \( X \) — случайная величина, а \( x \) — конкретное значение.\\
{\bf - Для непрерывных случайных величин:} Функция плотности вероятности \( f(x) \), такая что вероятность того, что случайная величина \( X \) лежит в интервале \([a, b]\) вычисляется как интеграл от функции плотности:
     \[
     P(a \leq X \leq b) = \int_{a}^{b} f(x) \, dx
     \]
\end{frame}
%------------------------------------------------

\begin{frame}
\frametitle{Основные понятия}
{\bf 3. Функция распределения (CDF, Cumulative Distribution Function):} Функция \( F(x) \) определяет вероятность того, что случайная величина \( X \) примет значение меньше или равно \( x \).\\
   - Для дискретных случайных величин:
     \[
     F(x) = P(X \leq x)
     \]
   - Для непрерывных случайных величин:
     \[
     F(x) = \int_{-\infty}^{x} f(t) \, dt
     \]
где $f(x)$ - функция плотности распределения случайной величины (probability density function).
\end{frame}
%------------------------------------------------

\begin{frame}
\frametitle{Свойства распределений}
{\bf Математическое ожидание (среднее)}
\newline\\
{\bf Для дискретных величин:}
     \[
     E(X) = \sum_{x} x \cdot P(X = x)
     \]
{\bf Для непрерывных величин:}
     \[
     E(X) = \int_{-\infty}^{\infty} x \cdot f(x) \, dx
     \]
\end{frame}
%------------------------------------------------

\begin{frame}
\frametitle{Свойства распределений}
{\bf Дисперсия}
\newline\\
{\bf Для дискретных величин:}
     \[
     \text{Var}(X) = E(X^2) - (E(X))^2
     \]
{\bf Для непрерывных величин:}
     \[
     \text{Var}(X) = \int_{-\infty}^{\infty} (x - E(X))^2 \cdot f(x) \, dx
     \]
\end{frame}
%------------------------------------------------

\begin{frame}
\frametitle{Свойства распределений}
{\bf Функция распределения (CDF)} указывает вероятность того, что случайная величина примет значение меньше или равно заданному значению.
\end{frame}
%------------------------------------------------

\begin{frame}
\frametitle{Дискретные распределения}
{\bf Дискретные распределения} описывают вероятности, связанные с дискретными случайными величинами, которые могут принимать конечное или счётное количество значений. Рассмотрим следующие основные дискретные распределения: равномерное, биномиальное, геометрическое, гипергеометрическое и распределение Пуассона.
\end{frame}
%------------------------------------------------

\begin{frame}
\frametitle{Дискретные распределения}
{\bf Равномерное распределение} описывает случайную величину, которая может принимать \( n \) различных значений, каждое из которых имеет одинаковую вероятность. Это распределение используется, когда все исходы равновероятны.\\
{\bf Функция вероятностей:} Если случайная величина \( X \) принимает значения \( x_1, x_2, \ldots, x_n \), то вероятность каждого значения:
  \[
  P(X = x_i) = \frac{1}{n}
  \]
где \( i = 1, 2, \ldots, n \).\\
{\bf Математическое ожидание (среднее):}
  \[
  E(X) = \frac{1}{n} \sum_{i=1}^{n} x_i
  \]
{\bf Дисперсия:}
  \[
  \text{Var}(X) = \frac{1}{n} \sum_{i=1}^{n} (x_i - E(X))^2
  \]
\end{frame}
%------------------------------------------------

\begin{frame}
\frametitle{Дискретные распределения}
{\bf Равномерное распределение}\\
{\bf Пример:} Если мы бросаем симметричную игральную кость, то вероятности выпадения каждого числа от 1 до 6 равны \( \dfrac{1}{6} \), математическое ожидание \( E(X) \) равно 3.5, а дисперсия \( \text{Var}(X) \) равна 2.92.
\end{frame}
%------------------------------------------------

\begin{frame}
\frametitle{Дискретные распределения}
{\bf Биномиальное распределение} моделирует количество успехов в \( n \) независимых испытаниях, каждое из которых имеет вероятность успеха \( p \).\\
{\bf Функция вероятностей:} Для случайной величины \( X \), которая обозначает количество успехов:
  \[
  P(X = k) = \binom{n}{k} p^k (1 - p)^{n - k}
  \]
где:
\\  - \( \dbinom{n}{k} \) — биномиальный коэффициент, равный \( \dfrac{n!}{k!(n - k)!} \),
\\  - \( k \) — число успехов,
\\  - \( n \) — число испытаний,
\\  - \( p \) — вероятность успеха в каждом испытании.
\end{frame}
%------------------------------------------------

\begin{frame}
\frametitle{Дискретные распределения}
{\bf Биномиальное распределение}\\
{\bf Математическое ожидание (среднее):}
  \[
  E(X) = n \cdot p
  \]
{\bf Дисперсия:}
  \[
  \text{Var}(X) = n \cdot p \cdot (1 - p)
  \]
{\bf Пример:} Если вы бросаете монету 10 раз, и вероятность выпадения орла в каждом броске равна 0.5, вероятность получить ровно 6 орлов:
  \[
  P(X = 6) = \binom{10}{6} (0.5)^6 (0.5)^{4} = \frac{10!}{6!4!} \cdot (0.5)^{10} \approx 0.205
  \]
\end{frame}
%------------------------------------------------

\begin{frame}
\frametitle{Дискретные распределения}
{\bf Геометрическое распределение} описывает количество испытаний до первого успеха, где вероятность успеха в каждом испытании постоянна и равна \( p \).\\
{\bf Функция вероятностей:} Для случайной величины \( X \), которая обозначает количество испытаний до первого успеха:
  \[
  P(X = k) = (1 - p)^{k - 1} p
  \]
где \( k \) — номер испытания, на котором произошел первый успех.\\
{\bf Математическое ожидание (среднее):}
  \[
  E(X) = \frac{1}{p}
  \]
{\bf Дисперсия:}
  \[
  \text{Var}(X) = \frac{1 - p}{p^2}
  \]
\end{frame}
%------------------------------------------------

\begin{frame}
\frametitle{Дискретные распределения}
{\bf Геометрическое распределение}\\
{\bf Пример:} Если вероятность успеха в каждом испытании равна 0.3, вероятность того, что первый успех произойдет на 4-м испытании:
  \[
  P(X = 4) = (1 - 0.3)^{4 - 1} \cdot 0.3 = 0.7^3 \cdot 0.3 \approx 0.1029
  \]
\end{frame}
%------------------------------------------------

\begin{frame}
\frametitle{Дискретные распределения}
{\bf Гипергеометрическое распределение} моделирует количество успехов в выборке без замены из конечного популяционного набора, где в популяции \( N \) элементов, из которых \( K \) являются успехами.\\
{\bf Функция вероятностей:} Для случайной величины \( X \), которая обозначает количество успехов \( k \) в выборке из \( n \) элементов:
  \[
  P(X = k) = \dfrac{\binom{K}{k} \binom{N - K}{n - k}}{\binom{N}{n}}
  \]
  где:
\\  - \( \dbinom{K}{k} \) — число способов выбрать \( k \) успехов из \( K \),
\\ - \( \dbinom{N - K}{n - k} \) — число способов выбрать \( n - k \) неуспехов из оставшихся \( N - K \),
\\ - \( \dbinom{N}{n} \) — число способов выбрать \( n \) элементов из \( N \).\\
\end{frame}
%------------------------------------------------

\begin{frame}
\frametitle{Дискретные распределения}
{\bf Гипергеометрическое распределение}\\
{\bf Математическое ожидание (среднее):}
  \[
  E(X) = n \cdot \frac{K}{N}
  \]
{\bf Дисперсия:}
  \[
  \text{Var}(X) = n \cdot \frac{K}{N} \cdot \frac{N - K}{N} \cdot \frac{N - n}{N - 1}
  \]
{\bf Пример:} В коробке 20 шаров, из которых 7 красные и 13 синие. Если мы выбираем 5 шаров без замены, вероятность того, что из выбранных 5 шаров ровно 3 будут красными:
  \[
  P(X = 3) = \frac{\binom{7}{3} \binom{13}{2}}{\binom{20}{5}} \approx 0.176
  \]
\end{frame}
%------------------------------------------------

\begin{frame}
\frametitle{Дискретные распределения}
{\bf Распределение Пуассона} моделирует количество событий, происходящих в фиксированном интервале времени или пространства, если события происходят с постоянной средней частотой и независимо друг от друга.\\
{\bf Функция вероятностей:} Для случайной величины \( X \), которая обозначает количество событий:
  \[
  P(X = k) = \frac{\lambda^k e^{-\lambda}}{k!}
  \]
где \( \lambda \) — среднее число событий в интервале, \( k \) — число событий.\\
{\bf Математическое ожидание (среднее):}
  \[
  E(X) = \lambda
  \]
{\bf Дисперсия:}
  \[
  \text{Var}(X) = \lambda
  \]
\end{frame}
%------------------------------------------------

\begin{frame}
\frametitle{Дискретные распределения}
{\bf Распределение Пуассона}\\
{\bf Пример:} Если в среднем в ресторане за час приходит 3 клиента, вероятность того, что за следующий час придет ровно 5 клиентов:
  \[
  P(X = 5) = \frac{3^5 e^{-3}}{5!} \approx 0.1008
  \]
\end{frame}

\end{document}