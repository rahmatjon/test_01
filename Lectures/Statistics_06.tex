\documentclass[aspectratio=169]{beamer}
%\documentclass{beamer}
\usepackage[utf8]{inputenc}
\usepackage[T2A]{fontenc}
\usepackage[russian]{babel}
\usepackage{cmap}
\usetheme{Boadilla}

\newcommand{\Ki}{\mathscr{K}}
\newcommand{\D}{\mathscr{D}}
\newcommand{\be}{\begin{equation}}
\newcommand{\ee}{\end{equation}}
\newcommand{\bes}{\begin{equation*}}
\newcommand{\ees}{\end{equation*}}

\makeatletter
\setbeamertemplate{footline}{%
  \leavevmode%
  \hbox{%
    \begin{beamercolorbox}[wd=.92\paperwidth,ht=2.25ex,dp=1ex,center]{title in head/foot}%
      \usebeamerfont{title in head/foot}\insertshorttitle
    \end{beamercolorbox}%
  }%
  \begin{beamercolorbox}[wd=.08\paperwidth,ht=2.25ex,dp=1ex,right]{date in head/foot}%
    \usebeamerfont{date in head/foot}%
    \usebeamertemplate{page number in head/foot}%
    \hspace*{2ex} 
  \end{beamercolorbox}
  \vskip0pt%
}
\makeatother

\title{\bf Урок 6. Распределения случайных величин (часть 2)}
\author{{\bf Хакимов Р.И. + ChatGPT}}
 \date[\today]{}

\begin{document}
\begin{frame}
\titlepage
\end{frame}
%------------------------------------------------

\begin{frame}
\frametitle{Непрерывные распределения}
{\bf Непрерывные распределения} описывают случайные величины, которые могут принимать любое значение в непрерывном интервале. Рассмотрим основные непрерывные распределения: равномерное, нормальное, экспоненциальное, гамма и распределение Коши.
\end{frame}
%------------------------------------------------

\begin{frame}
\frametitle{Непрерывные распределения}
{\bf Равномерное распределение} распределение описывает случайную величину, которая имеет равные вероятности для всех значений в интервале \([a, b]\).\\
{\bf Функция плотности вероятности:}
  \[
  f(x) = \frac{1}{b - a}
  \]
для \( a \leq x \leq b \), и \( f(x) = 0 \) в других случаях.\\
{\bf Пример:} Если число случайно выбирается из интервала от 0 до 1, его распределение будет равномерным.
\end{frame}
%------------------------------------------------

\begin{frame}
\frametitle{Непрерывные распределения}
{\bf Нормальное распределение}, или распределение Гаусса, описывает многие естественные явления и статистические измерения. Оно характеризуется двумя параметрами: средним \( \mu \) и стандартным отклонением \( \sigma \).\\
{\bf Функция плотности вероятности:}
  \[
  f(x) = \dfrac{1}{\sigma\sqrt{2 \pi }} e^{-\dfrac{(x - \mu)^2}{2 \sigma^2}}
  \]
где \( \mu \) — среднее значение, \( \sigma \) — стандартное отклонение.\\
{\bf Пример:} Рост людей в популяции часто подчиняется нормальному распределению. Если средний рост составляет 170 см, а стандартное отклонение — 10 см, можно использовать нормальное распределение для оценки вероятности того, что рост случайно выбранного человека будет в пределах определенного интервала.
\end{frame}
%------------------------------------------------

\begin{frame}
\frametitle{Непрерывные распределения}
{\bf Экспоненциальное распределение} описывает время между событиями в процессе с постоянной средней частотой. Оно является частным случаем гамма-распределения с параметром \( \alpha = 1 \).\\
{\bf Функция плотности вероятности:}
  \[
  f(x) = \lambda e^{-\lambda x}
  \]
где \( \lambda \) — параметр распределения (обратный среднему времени до события), \( x \geq 0 \).\\
{\bf Пример:} Время ожидания следующего автобуса в автобусной остановке с постоянным интервалом прибытия можно моделировать с помощью экспоненциального распределения.
\end{frame}
%------------------------------------------------

\begin{frame}
\frametitle{Непрерывные распределения}
{\bf Экспоненциальное распределение}\\
{\bf Задача:} Автобусы на остановку приезжают в среднем каждые 15 минут. Какова вероятность того, что следующий автобус придет через 10 минут или меньше?\\
{\bf Решение:}\\
1. Параметр $\lambda$ -- это интенсивность процесса (количество автобусов в единицу времени). Если автобус приходит в среднем каждые 15 минут, то $\lambda$ будет равен $\dfrac{1}{15}$ автобусов в минуту.\\
2. Экспоненциальное распределение описывает время между событиями, его функция распределения для времени $t$ выглядит так:\\
$$F(t) = 1 - e^{-\lambda t}$$
Это вероятность того, что событие (прибытие автобуса) произойдет в течение времени $t$.
\end{frame}
%------------------------------------------------

\begin{frame}
\frametitle{Непрерывные распределения}
{\bf Экспоненциальное распределение}\\
{\bf Решение:}\\
3. Подставляем значения: Нам нужно найти вероятность того, что автобус придет через 10 минут или меньше. Для этого подставим $\lambda = \dfrac{1}{15}$ и $t = 10$ минут в формулу:
$$F(10)=1 - e^{-\frac{1}{15}\cdot 10} = 1 - e^{-\frac{2}{3}} \approx 1 - 0.5134 \approx 0.4866$$
4. Результат: Вероятность того, что автобус придет в течение 10 минут, составляет примерно $48.66\%$.
\end{frame}
%------------------------------------------------

\begin{frame}
\frametitle{Непрерывные распределения}
{\bf Гамма распределение} обобщает экспоненциальное распределение и моделирует время до \( k \)-го события в процессе с постоянной средней частотой. Имеет два параметра: \( \alpha \) (форма) и \( \beta \) (масштаб).\\
{\bf Функция плотности вероятности:}
  \[
  f(x) = \frac{x^{\alpha - 1} e^{-x / \beta}}{\beta^{\alpha} \Gamma(\alpha)}
  \]
где \( \Gamma(\alpha) \) — функция Гамма, \( \alpha \) — параметр формы, \( \beta \) — параметр масштаба, \( x \geq 0 \).\\
$\Gamma (z)=\int \limits _{0}^{+\infty }t^{z-1}e^{-t}\,dt,\quad z\in \mathbb {C} ,\quad  Re(z)>0$ - является обощением понятия факториала.\\
{\bf Пример:} Если время до поломки устройства можно рассматривать как сумму времени до поломки нескольких компонентов, это может быть смоделировано с помощью гамма распределения.
\end{frame}
%------------------------------------------------

\begin{frame}
\frametitle{Непрерывные распределения}
{\bf Распределение Коши} имеет "тяжелые хвосты" и используется для моделирования явлений, где экстремальные значения более вероятны, чем в нормальном распределении. Оно характеризуется параметрами: медианой \( x_0 \) и масштабом \( \gamma \).\\
{\bf Функция плотности вероятности:}
  \[
  f(x) = \frac{1}{\pi \gamma \left[ 1 + \left( \frac{x - x_0}{\gamma} \right)^2 \right]}
  \]
где \( x_0 \) — медиана, \( \gamma \) — параметр масштаба.\\
{\bf Пример:} Распределение Коши может быть использовано для моделирования финансовых данных, где резкие колебания цен встречаются чаще, чем предсказывается нормальным распределением.
\end{frame}
%------------------------------------------------

\begin{frame}
\frametitle{Свойства непрерывных распределений}
{\bf Математическое ожидание и дисперсия:}
\newline\\
{\bf Нормальное распределение:} Среднее значение \( \mu \) и дисперсия \( \sigma^2 \).
\newline\\
{\bf Экспоненциальное распределение:} Среднее \( \dfrac{1}{\lambda} \) и дисперсия \( \dfrac{1}{\lambda^2} \).
\newline\\
{\bf Гамма распределение:} Среднее \( \alpha \beta \) и дисперсия \( \alpha \beta^2 \).
\newline\\
{\bf Равномерное распределение:} Среднее \( \dfrac{a + b}{2} \) и дисперсия \( \dfrac{(b - a)^2}{12} \).
\newline\\
{\bf Распределение Коши:} Не имеет конечного математического ожидания и дисперсии.
\end{frame}
%------------------------------------------------

\begin{frame}
\frametitle{Свойства непрерывных распределений}
{\bf Функция распределения (CDF):}
\newline\\
{\bf Нормальное распределение:} $F(x)={\dfrac {1}{\sqrt {2\pi }}}\int \limits _{-\infty }^{x}e^{-t^{2}/2}\,dt$.
\newline\\
{\bf Экспоненциальное распределение:} $F(x) = 1 - e^{-\lambda x}$.
\newline\\
{\bf Гамма распределение:} $F(x) = \frac{\gamma(\alpha, x / \beta)}{\Gamma(\alpha)}$.
\newline\\
{\bf Равномерное распределение:} $F(x) = \dfrac{x - a}{b - a} \text{ для } a \leq x \leq b$.
\newline\\
{\bf Распределение Коши:} $F(x) = \dfrac{1}{\pi} \arctan\left(\frac{x - x_0}{\gamma}\right) + \frac{1}{2}$.
\end{frame}

\end{document}