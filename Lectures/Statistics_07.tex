\documentclass[aspectratio=169]{beamer}
%\documentclass{beamer}
\usepackage[utf8]{inputenc}
\usepackage[T2A]{fontenc}
\usepackage[russian]{babel}
\usepackage{cmap}
\usetheme{Boadilla}

\newcommand{\Ki}{\mathscr{K}}
\newcommand{\D}{\mathscr{D}}
\newcommand{\be}{\begin{equation}}
\newcommand{\ee}{\end{equation}}
\newcommand{\bes}{\begin{equation*}}
\newcommand{\ees}{\end{equation*}}

\makeatletter
\setbeamertemplate{footline}{%
  \leavevmode%
  \hbox{%
    \begin{beamercolorbox}[wd=.92\paperwidth,ht=2.25ex,dp=1ex,center]{title in head/foot}%
      \usebeamerfont{title in head/foot}\insertshorttitle
    \end{beamercolorbox}%
  }%
  \begin{beamercolorbox}[wd=.08\paperwidth,ht=2.25ex,dp=1ex,right]{date in head/foot}%
    \usebeamerfont{date in head/foot}%
    \usebeamertemplate{page number in head/foot}%
    \hspace*{2ex} 
  \end{beamercolorbox}
  \vskip0pt%
}
\makeatother

\title{\bf Урок 7. Статистика вывода (часть 1)}
\author{{\bf Хакимов Р.И. + ChatGPT}}
 \date[\today]{}

\begin{document}
\begin{frame}
\titlepage
\end{frame}
%------------------------------------------------

\begin{frame}
\frametitle{Оценка параметров: выборочное среднее и выборочная дисперсия}
В статистике вывода оценка параметров является важным аспектом для обобщения результатов выборки на генеральную совокупность. Два ключевых параметра, которые часто оцениваются, это выборочное среднее и выборочная дисперсия.
\end{frame}
%------------------------------------------------

\begin{frame}
\frametitle{Оценка параметров: выборочное среднее и выборочная дисперсия}
{\bf Выборочное среднее (Sample Mean)} является оценкой математического ожидания (среднего значения) генеральной совокупности на основе данных выборки.\\
{\bf Формула:}
  \[
  \bar{X} = \frac{1}{n} \sum_{i=1}^{n} X_i
  \]
где: \( \bar{X} \) — выборочное среднее, \( n \) — размер выборки, \( X_i \) — значение \( i \)-го наблюдения в выборке.\\
{\bf Свойства:}\\
{\bf Смещенность:} Выборочное среднее является несмещенной оценкой истинного среднего значения генеральной совокупности.\\
{\bf Эффективность:} С увеличением размера выборки выборочное среднее становится более точным.
\end{frame}
%------------------------------------------------

\begin{frame}
\frametitle{Оценка параметров: выборочное среднее и выборочная дисперсия}
{\bf Выборочное среднее (Sample Mean)}\\
{\bf Пример:} Если в выборке из 10 людей рост составляет: 160, 165, 170, 155, 180, 175, 160, 165, 170, 155 см, то выборочное среднее роста будет:
  \[
  \bar{X} = \frac{160 + 165 + 170 + 155 + 180 + 175 + 160 + 165 + 170 + 155}{10} = 167
  \]
\end{frame}
%------------------------------------------------

\begin{frame}
\frametitle{Оценка параметров: выборочное среднее и выборочная дисперсия}
{\bf Выборочная дисперсия (Sample Variance)} оценивает разброс данных вокруг выборочного среднего и является оценкой дисперсии генеральной совокупности.\\
{\bf Формула:}
  \[
  S^2 = \frac{1}{n - 1} \sum_{i=1}^{n} (X_i - \bar{X})^2
  \]
где: \( S^2 \) — выборочная дисперсия, \( \bar{X} \) — выборочное среднее, \( n \) — размер выборки, \( X_i \) — значение \( i \)-го наблюдения в выборке.\\
{\bf Свойства:}\\
{\bf Смещенность:} Выборочная дисперсия является несмещенной оценкой дисперсии генеральной совокупности, благодаря делению на \( n - 1 \) вместо \( n \). Этот фактор корректирует смещение, которое возникает из-за оценки средней величины на основе выборки.\\
{\bf Эффективность:} С увеличением размера выборки выборочная дисперсия становится более точной.
\end{frame}
%------------------------------------------------

\begin{frame}
\frametitle{Оценка параметров: выборочное среднее и выборочная дисперсия}
{\bf Выборочная дисперсия (Sample Variance)}\\
{\bf Пример:} Для той же выборки, что и в предыдущем примере (рост): 160, 165, 170, 155, 180, 175, 160, 165, 170, 155 см, выборочная дисперсия будет:
\begin{multline*}
S^2 = \frac{1}{10 - 1}\times((160 - 167)^2 + (165 - 167)^2 + (170 - 167)^2 + (155 - 167)^2 + (180 - 167)^2 +\\ (175 - 167)^2 + (160 - 167)^2 + (165 - 167)^2 + (170 - 167)^2 + (155 - 167)^2) = \frac{2604}{9} \approx 289.33
\end{multline*}
\end{frame}
%------------------------------------------------

\begin{frame}
\frametitle{Оценка параметров: выборочное среднее и выборочная дисперсия}
{\bf Выборочное стандартное отклонение (Sample Standard Deviation)} - это квадратный корень из выборочной дисперсии и предоставляет меру разброса в тех же единицах, что и исходные данные.
  \[
  S = \sqrt{S^2}
  \]
\end{frame}
%------------------------------------------------

\begin{frame}
\frametitle{Оценка параметров: выборочное среднее и выборочная дисперсия}
{\bf Надежность оценок:} Чем больше размер выборки, тем надежнее (меньше ошибка) оценки параметров. Также важно учитывать уровень доверия и строить доверительные интервалы для более полного понимания параметров генеральной совокупности.
\newline\\
Эти оценки являются основой для проведения статистических тестов и построения моделей, которые помогают делать выводы о генеральной совокупности на основе выборки данных.
\end{frame}
%------------------------------------------------

\begin{frame}
\frametitle{Введение в тестирование гипотез}
{\bf Тестирование гипотез} — это статистический метод, используемый для проверки предположений (гипотез) о параметрах генеральной совокупности на основе выборочных данных. Этот процесс помогает определить, насколько убедительными являются результаты исследования, и позволяет сделать выводы о популяции, исходя из данных выборки.
\end{frame}
%------------------------------------------------

\begin{frame}
\frametitle{Основные концепции тестирования гипотез}
{\bf Гипотеза:} Предположение о параметре генеральной совокупности, которое проверяется на основе данных выборки.
\newline\\  
{\bf Нулевая гипотеза ($H_0$):} Основное предположение, которое мы тестируем. Обычно это гипотеза о том, что нет эффекта или различий. Например, "средний доход мужчин и женщин одинаков".
\newline\\  
{\bf Альтернативная гипотеза ($H_1$ или $H_a$):} Гипотеза, которая предполагает наличие эффекта или различий. Это гипотеза, которая рассматривается в случае, если нулевая гипотеза отвергнута. Например, "средний доход мужчин и женщин различен".
\end{frame}
%------------------------------------------------

\begin{frame}
\frametitle{Основные концепции тестирования гипотез}
В математической статистике уровень значимости (\(\alpha\)) и уровень доверия (\(1 - \alpha\)) связаны следующим образом:
\newline\\  
{\bf Уровень значимости ($\alpha$):} – это вероятность ошибки первого рода ($\alpha$), то есть вероятность отклонения нулевой гипотезы, когда она на самом деле верна. Обычно выбирается значение 0.05 или 0.01.
\newline\\  
{\bf Уровень доверия (\(1 - \alpha\))} – это вероятность того, что истинное значение параметра действительно попадает в доверительный интервал.\\
Таким образом, они дополняют друг друга до 1:
\[
\text{Уровень доверия} = 1 - \text{Уровень значимости}
\]
или
\[
\text{Уровень значимости} = 1 - \text{Уровень доверия}
\]
\end{frame}
%------------------------------------------------

\begin{frame}
\frametitle{Основные концепции тестирования гипотез}
Например, если уровень значимости \(\alpha = 0.05\) (5\%), то уровень доверия составляет \(1 - 0.05 = 0.95\) (95\%). Это означает, что с вероятностью 95\% истинное значение параметра будет находиться в рассчитанном доверительном интервале.
\newline\\
Практическое значение:\\
- В проверке гипотез \(\alpha\) определяет вероятность ложного отклонения нулевой гипотезы. Чем меньше \(\alpha\), тем более строгий критерий проверки.\\
- В доверительных интервалах \(1 - \alpha\) показывает степень уверенности в том, что интервал охватывает истинное значение параметра.
\end{frame}
%------------------------------------------------

\begin{frame}
\frametitle{Односторонний и двусторонний критерии}
В математической статистике уровень значимости (\(\alpha\)) связан с хвостами нормального распределения, так как он определяет области, где отклоняется нулевая гипотеза.
\newline\\
{\bf Односторонний критерий (правосторонний или левосторонний)}\\
- Если тест правосторонний, то весь уровень значимости \(\alpha\) находится в правом хвосте нормального распределения. Например, если \(\alpha = 0.05\), то 5\% площади распределения оказывается в правом хвосте.\\
- Если тест левосторонний, то уровень значимости \(\alpha\) находится в левом хвосте.
Например, если уровень значимости \(\alpha = 0.05\) (5\%), то уровень доверия составляет \(1 - 0.05 = 0.95\) (95\%). Это означает, что с вероятностью 95\% истинное значение параметра будет находиться в рассчитанном доверительном интервале.
\end{frame}
%------------------------------------------------

\begin{frame}
\frametitle{Односторонний и двусторонний критерии}
{\bf Двусторонний критерий}\\
- В этом случае уровень значимости \(\alpha\) делится поровну между двумя хвостами распределения. Например, при \(\alpha = 0.05\) по 2.5\% (0.025) площади оказывается в левом и правом хвостах.  
\newline\\
{\bf Графическая интерпретация}\\
- В правостороннем тесте \(\alpha\) находится справа от критического значения \(Z_{\alpha}\).\\
- В левостороннем тесте \(\alpha\) находится слева от критического значения \(Z_{\alpha}\).\\
- В двустороннем тесте область \(\alpha/2\) расположена в обоих хвостах распределения.  
\newline\\
{\bf Связь с уровнем доверия}\\
Так как уровень доверия \(1 - \alpha\) охватывает основную часть нормального распределения (обычно среднюю симметричную область), хвосты представляют исключительные, маловероятные значения.
\end{frame}
%------------------------------------------------

\begin{frame}
\frametitle{Основные концепции тестирования гипотез}
{\bf Ошибка первого рода ($\alpha$):} Ошибка, при которой нулевая гипотеза отвергается, хотя она верна.\\
Ошибка первого рода важна для понимания, так как её последствия могут быть серьезными, и она учитывается при планировании экспериментов и интерпретации их результатов.
\newline\\
{\bf Ошибка второго рода ($\beta$):} Ошибка, при которой нулевая гипотеза не отвергается, хотя альтернативная гипотеза верна.\\
Ошибка второго рода особенно критична, когда важно не пропустить реальные эффекты или различия, например, в медицине, правосудии или научных исследованиях.
\end{frame}
%------------------------------------------------

\begin{frame}
\frametitle{Ошибка первого рода. Примеры}
{\bf 1. Медицинское исследование}\\
\quad {\it Сценарий:} Проводится тест на наличие заболевания у человека. Нулевая гипотеза -- пациент здоров.\\
\quad {\it Ошибка первого рода:} Тест показывает, что пациент болен (положительный результат), хотя он здоров.\\
\quad {\it Последствия:} Пациент может получить ненужное лечение, что может повлечь за собой побочные эффекты или затраты.
\newline\\
{\bf 2. Научное исследование}\\
\quad {\it Сценарий:} В эксперименте проверяется гипотеза о наличии нового физического явления. Нулевая гипотеза -- нового явления нет.\\
\quad {\it Ошибка первого рода:} Ученые сообщают о наличии нового явления, хотя его нет, и различия в данных вызваны случайностью.\\
\quad {\it Последствия:} Это может привести к ложным научным открытиям и потерям времени на дальнейшие исследования.
\end{frame}
%------------------------------------------------

\begin{frame}
\frametitle{Ошибка первого рода. Примеры}
{\bf 3. Клинические испытания нового лекарства}\\
\quad {\it Сценарий:} Проводится исследование эффективности нового лекарства против заболевания. Нулевая гипотеза -- лекарство не более эффективно, чем плацебо.\\
\quad {\it Ошибка первого рода:} Исследование показывает, что лекарство эффективно, хотя на самом деле оно не лучше плацебо.\\
\quad {\it Последствия:} Лекарство может выйти на рынок и применяться, несмотря на отсутствие реального эффекта.
\newline\\
{\bf  4. Судебное разбирательство}\\
\quad {\it Сценарий:} В суде нулевая гипотеза -- обвиняемый невиновен (презумпция невиновности).\\
\quad {\it Ошибка первого рода:} Суд выносит обвинительный приговор (решает, что обвиняемый виновен), хотя он невиновен.\\
\quad {\it Последствия:} Невиновный человек может быть осужден и понести наказание за преступление, которого он не совершал.
\end{frame}
%------------------------------------------------

\begin{frame}
\frametitle{Ошибка второго рода. Примеры}
{\bf 1. Медицинское исследование}\\
\quad {\it Сценарий:} Проводится тест на наличие заболевания у человека. Нулевая гипотеза -- пациент здоров.\\
\quad {\it Ошибка второго рода:} Тест показывает, что пациент здоров, хотя на самом деле он болен.\\
\quad {\it Последствия:} Это может привести к отсутствию своевременного лечения, что ухудшает состояние пациента и усложняет лечение в будущем.
\newline\\
{\bf 2. Клинические испытания нового лекарства}\\
\quad {\it Сценарий:} Проводится исследование эффективности нового лекарства против заболевания. Нулевая гипотеза -- лекарство не более эффективно, чем плацебо.\\
\quad {\it Ошибка второго рода:} Исследование не находит доказательств эффективности лекарства, хотя на самом деле оно работает.\\
\quad {\it Последствия:} Эффективное лекарство может быть отклонено, и оно не будет использовано для лечения пациентов, которые могли бы от него выиграть.
\end{frame}
%------------------------------------------------

\begin{frame}
\frametitle{Ошибка второго рода. Примеры}
{\bf  3. Судебное разбирательство}\\
\quad {\it Сценарий:} В суде нулевая гипотеза -- обвиняемый невиновен (презумпция невиновности).\\
\quad {\it Ошибка второго рода:} Суд оправдывает обвиняемого (принимает его невиновность), хотя он на самом деле виновен.\\
\quad {\it Последствия:} Виновный человек остается на свободе и может совершить новые преступления.
\newline\\
{\bf  4. Эксперименты в маркетинге}\\
\quad {\it Сценарий:} Компания тестирует новую маркетинговую стратегию для увеличения продаж. Нулевая гипотеза — новая стратегия не улучшит продажи.\\
\quad {\it Ошибка второго рода:} Эксперимент не показывает значительного роста продаж, хотя на самом деле стратегия эффективна.\\
\quad {\it Последствия:} Компания может отказаться от успешной маркетинговой стратегии, теряя потенциальные доходы.
\end{frame}
%------------------------------------------------

\begin{frame}
\frametitle{Доверительные интервалы}
{\bf Доверительные интервалы} в статистике вывода используются для оценки параметров генеральной совокупности на основе данных выборки и предоставляют диапазон, в котором с определённой вероятностью находится истинное значение параметра. Они позволяют учесть неопределённость, связанную с выборкой, и дают представление о надёжности оценок.
\end{frame}
%------------------------------------------------

\begin{frame}
\frametitle{Доверительные интервалы}
{\bf Основные понятия}
\newline\\
{\bf Доверительный интервал (Confidence interval):} Это интервал значений, в котором с определённой вероятностью (уровень доверия) находится истинное значение параметра генеральной совокупности.\\
{\bf Уровень доверия (Confidence Level):} Это вероятность того, что доверительный интервал содержит истинное значение параметра. Обычно выбирается 90\%, 95\% или 99\%.
\end{frame}
%------------------------------------------------

\begin{frame}
\frametitle{Формулы для доверительных интервалов}
{\bf Доверительный интервал для среднего значения (при известной дисперсии)}\\
Если дисперсия генеральной совокупности известна и распределение данных нормально, доверительный интервал для среднего значения рассчитывается как:
\[
\bar{X} \pm Z_{\alpha/2} \cdot \frac{\sigma}{\sqrt{n}}
\]
где:\\
- \( \bar{X} \) — выборочное среднее,\\
- \( Z_{\alpha/2} \) — критическое значение Z для выбранного уровня доверия,\\
- \( \sigma \) — стандартное отклонение генеральной совокупности (или выборки, если это точная оценка),\\
- \( n \) — размер выборки.
\end{frame}
%------------------------------------------------

\begin{frame}
\frametitle{Формулы для доверительных интервалов}
{\bf Доверительный интервал для среднего значения (при неизвестной дисперсии)}\\
Если дисперсия генеральной совокупности неизвестна и распределение данных нормально, используется распределение Стьюдента (t-распределение):
\[
\bar{X} \pm t_{\alpha/2, n-1} \cdot \frac{S}{\sqrt{n}}
\]
где:\\
- \( t_{\alpha/2, n-1} \) — критическое значение t для выбранного уровня доверия и \( n-1 \) степеней свободы,\\
- \( S \) — выборочное стандартное отклонение.
\end{frame}
%------------------------------------------------

\begin{frame}
\frametitle{Формулы для доверительных интервалов}
{\bf Доверительный интервал для дисперсии}\\
Если необходимо оценить дисперсию генеральной совокупности, доверительный интервал для дисперсии рассчитывается как (используется $\chi^2$-распределение):
\[
\left( \frac{(n-1)S^2}{\chi^2_{\alpha/2, n-1}}, \frac{(n-1)S^2}{\chi^2_{1-\alpha/2, n-1}} \right)
\]
где:\\
- \( S^2 \) — выборочная дисперсия,\\
- \( \chi^2_{\alpha/2, n-1} \) и \( \chi^2_{1-\alpha/2, n-1} \) — критические значения \(\chi^2\) для выбранного уровня доверия и \( n-1 \) степеней свободы.
\end{frame}
%------------------------------------------------

\begin{frame}
\frametitle{Формулы для доверительных интервалов}
{\bf Доверительный интервал для пропорции}\\
Если оценка касается пропорции, доверительный интервал рассчитывается как:
\[
\hat{p} \pm Z_{\alpha/2} \cdot \sqrt{\frac{\hat{p}(1 - \hat{p})}{n}}
\]
где:\\
- \( \hat{p} \) — выборочная пропорция,\\
- \(n\) — размер выборки,\\
- \(\alpha\) — уровень значимости,\\
- \( Z_{\alpha/2} \) — критическое значение Z для выбранного уровня доверия.
\end{frame}

\begin{frame}
\frametitle{Как найти критическое значение \(Z_{\alpha/2}\)?}
{\bf Критическое значение \(Z_{\alpha/2}\)} — это значение стандартного нормального распределения, которое отсекает \(\alpha/2\) от хвостов распределения.
\newline\\
Для некоторых стандартных уровней доверия:\\
- Для 90\% уровня доверия: \(Z_{\alpha/2} \approx 1.645\);\\
- Для 95\% уровня доверия: \(Z_{\alpha/2} \approx 1.96\);\\
- Для 99\% уровня доверия: \(Z_{\alpha/2} \approx 2.576\).
\end{frame}
%------------------------------------------------

\begin{frame}
\frametitle{Алгоритм поиска критического Z-значения в таблице}
При проведении {\bf двустороннего Z-теста} необходимо определить критическое значение \( Z_{\alpha/2} \), используя таблицу Z-значений (Z-score table).
\newline\\
{\bf 1. Определите уровень значимости (\(\alpha\))}\\
Например, если \(\alpha = 0.05\), то в двустороннем тесте по 2.5\% уходит в каждый хвост (так как \( \alpha/2 = 0.025 \)).  
\newline\\
{\bf 2. Найдите вероятность в центральной части таблицы}\\
- В таблице Z-значений даны вероятности слева от Z. \\
- Нам нужно найти вероятность, соответствующую \( 1 - \alpha/2 \). \\
- Например, если \(\alpha = 0.05\), то \( 1 - 0.025 = 0.975 \). 
\end{frame}
%------------------------------------------------

\begin{frame}
\frametitle{Алгоритм поиска критического Z-значения в таблице}
{\bf 3. Найдите ближайшее значение в таблице}\\
- В таблице Z-значений ищем вероятность, ближайшую к 0.975. \\
- Это соответствует \(Z \approx 1.96\). 
\newline\\
{\bf 4. Запишите критические значения}\\
- Так как тест двусторонний, критические значения:  
\[
Z_{\alpha/2} = \pm 1.96
\]
- Если расчет выполняется для другого \(\alpha\) (например, 0.01), то нужно искать 0.995, что дает \( Z \approx \pm 2.576 \).  
\end{frame}
%------------------------------------------------

\begin{frame}
\frametitle{Как найти критическое значение \(t_{\alpha/2, df}\)?}
Для {\bf двустороннего t-теста} критическое значение \( t_{\alpha/2, df} \) ищется в t-таблице (t-table).
\newline\\
{\bf Примеры для популярных уровней значимости:}\\
\begin{center}
\begin{tabular}{ |c|c|c| } 
 \hline
 df & \( t_{0.05/2} \) (95\% доверие) & \( t_{0.01/2} \) (99\% доверие)\\ 
 \hline
 10 & 2.228 & 3.169\\
 \hline
 20 & 2.086 & 2.845\\
 \hline
 30 & 2.042 & 2.750\\
 \hline
\end{tabular}
\end{center}
\end{frame}
%------------------------------------------------

\begin{frame}
\frametitle{Алгоритм поиска критического t-значения в таблице}
{\bf 1. Определите уровень значимости (\(\alpha\))}\\
Например, если \(\alpha = 0.05\), то в двустороннем тесте по 2.5\% уходит в каждый хвост (так как \( \alpha/2 = 0.025 \)).
\newline\\
{\bf 2. Определите число степеней свободы \( df \)}\\
- Для одновыборочного t-теста:
     \[
     df = n - 1
     \]
- Для двухвыборочного t-теста (независимые выборки, равные дисперсии):
     \[
     df = n_1 + n_2 - 2
     \]
- Для двухвыборочного t-теста (неравные дисперсии, Welch’s test):  
     \[
     df \approx \frac{(s_1^2 / n_1 + s_2^2 / n_2)^2}{\frac{(s_1^2 / n_1)^2}{n_1 - 1} + \frac{(s_2^2 / n_2)^2}{n_2 - 1}}
     \]
     где \( s_1^2 \) и \( s_2^2 \) — дисперсии выборок.  
\end{frame}
%------------------------------------------------

\begin{frame}
\frametitle{Алгоритм поиска критического t-значения в таблице}
{\bf 3. Ищем в t-таблице:}\\
- Находим число степеней свободы \( df \) в левом столбце.\\
- Находим уровень значимости \( \alpha/2 \) в верхней строке.\\
- На пересечении — критическое значение \( t_{\alpha/2, df} \).  
\end{frame}
%------------------------------------------------
\end{document}