\documentclass[aspectratio=169]{beamer}
%\documentclass{beamer}
\usepackage[utf8]{inputenc}
\usepackage[T2A]{fontenc}
\usepackage[russian]{babel}
\usepackage{cmap}
\usetheme{Boadilla}

\newcommand{\Ki}{\mathscr{K}}
\newcommand{\D}{\mathscr{D}}
\newcommand{\be}{\begin{equation}}
\newcommand{\ee}{\end{equation}}
\newcommand{\bes}{\begin{equation*}}
\newcommand{\ees}{\end{equation*}}

\makeatletter
\setbeamertemplate{footline}{%
  \leavevmode%
  \hbox{%
    \begin{beamercolorbox}[wd=.92\paperwidth,ht=2.25ex,dp=1ex,center]{title in head/foot}%
      \usebeamerfont{title in head/foot}\insertshorttitle
    \end{beamercolorbox}%
  }%
  \begin{beamercolorbox}[wd=.08\paperwidth,ht=2.25ex,dp=1ex,right]{date in head/foot}%
    \usebeamerfont{date in head/foot}%
    \usebeamertemplate{page number in head/foot}%
    \hspace*{2ex} 
  \end{beamercolorbox}
  \vskip0pt%
}
\makeatother

\title{\bf Урок 8. Статистика вывода (часть 2)}
\author{{\bf Хакимов Р.И. + ChatGPT}}
 \date[\today]{}

\begin{document}
\begin{frame}
\titlepage
\end{frame}
%------------------------------------------------

\begin{frame}
\frametitle{Примеры}
{\bf Доверительный интервал для среднего значения}\\
Предположим, вы измерили рост 25 случайных людей и получили выборочное среднее 170 см, а выборочное стандартное отклонение — 10 см. Вы хотите построить 95\% доверительный интервал для среднего роста.\\
- Критическое значение t для 95\% доверительного интервала и 24 степеней свободы примерно равно 2.064.\\
- Доверительный интервал рассчитывается как:
  \[
  170 \pm 2.064 \cdot \frac{10}{\sqrt{25}} = 170 \pm 4.128
  \]
{\bf Интервал:} \( (165.872, 174.128) \).
\end{frame}
%------------------------------------------------

\begin{frame}
\frametitle{Примеры}
{\bf Доверительный интервал для пропорции}\\
Если из 200 опрошенных человек 120 предпочитают определённый бренд, то выборочная пропорция \( \hat{p} = \frac{120}{200} = 0.6 \). Для 95\% доверительного интервала:\\
- Критическое значение Z для 95\% доверительного интервала примерно равно 1.96.\\
- Доверительный интервал рассчитывается как:
  \[
  0.6 \pm 1.96 \cdot \sqrt{\frac{0.6 \cdot (1 - 0.6)}{200}} = 0.6 \pm 0.068
  \]
{\bf Интервал:} \( (0.532, 0.668) \).
\end{frame}
%------------------------------------------------

\begin{frame}
\frametitle{Примеры}
{\bf Доверительный интервал для пропорции}\\
Если из 150 опрошенных человек 90 предпочитают определённый бренд, то выборочная пропорция \( \hat{p} = \frac{90}{150} = 0.6 \). Для уровня доверия 95\% (где \( Z_{0.025} \approx 1.96 \)):
  \[
  0.6 \pm 1.96 \cdot \sqrt{\frac{0.6 \cdot (1 - 0.6)}{150}} \approx 0.6 \pm 0.079
  \]
{\bf Интервал:} \( (0.521, 0.679) \).
\end{frame}
%------------------------------------------------

\begin{frame}
\frametitle{Примеры}
{\bf Доверительный интервал для среднего значения, когда дисперсия генеральной совокупности известна}\\
Если выборочное среднее \( \bar{X} = 50 \), известное стандартное отклонение \( \sigma = 5 \), размер выборки \( n = 30 \), и уровень доверия 95\% (где \( Z_{0.025} \approx 1.96 \)):
  \[
  50 \pm 1.96 \cdot \frac{5}{\sqrt{30}} \approx 50 \pm 1.79
  \]
{\bf Интервал:} \( (48.21, 51.79) \).
\end{frame}
%------------------------------------------------

\begin{frame}
\frametitle{Примеры}
{\bf Доверительный интервал для среднего значения, когда дисперсия генеральной совокупности неизвестна}\\
Если выборочное среднее \( \bar{X} = 50 \), выборочное стандартное отклонение \( S = 5 \), размер выборки \( n = 30 \), и уровень доверия 95% (где \( t_{0.025, 29} \approx 2.045 \)):
  \[
  50 \pm 2.045 \cdot \frac{5}{\sqrt{30}} \approx 50 \pm 1.87
  \]
{\bf Интервал:} \( (48.13, 51.87) \).
\end{frame}
%------------------------------------------------

\begin{frame}
\frametitle{Примеры}
{\bf Доверительный интервал для дисперсии}\\
Если выборочная дисперсия \( S^2 = 25 \), размер выборки \( n = 20 \), и уровень доверия 95\% (где \( \chi^2_{0.025, 19} \approx 32.85 \) и \( \chi^2_{0.975, 19} \approx 8.91 \)):
  \[
  \left( \frac{(20-1) \cdot 25}{32.85}, \frac{(20-1) \cdot 25}{8.91} \right) \approx (14.67, 52.92)
  \]
{\bf Интервал:} \( (14.67, 52.92) \).
\end{frame}
%------------------------------------------------

\begin{frame}
\frametitle{Методы построения доверительных интервалов}
{\bf Методы для больших выборок}\\
Для больших выборок (обычно \( n > 30 \)) можно использовать приближенные методы и нормальное распределение для большинства параметров. В таких случаях выборочное среднее и дисперсия близки к нормальному распределению из-за центральной предельной теоремы.\\
{\bf Методы для маленьких выборок}\\
Для маленьких выборок (обычно \( n \leq 30 \)) важно учитывать распределение данных и использовать t-распределение или другие подходящие распределения для построения доверительных интервалов.
\end{frame}
%------------------------------------------------

\begin{frame}
\frametitle{Методы построения доверительных интервалов}
{\bf Заключение}\\
Методы построения доверительных интервалов зависят от параметров популяции, характеристик выборки и условий, при которых собираются данные. Правильное применение этих методов позволяет оценить диапазон, в котором, с заданной вероятностью, находится истинное значение интересующего параметра, что способствует принятию обоснованных решений и интерпретации данных.
\end{frame}

\end{document}