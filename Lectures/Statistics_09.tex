\documentclass[aspectratio=169]{beamer}
%\documentclass{beamer}
\usepackage[utf8]{inputenc}
\usepackage[T2A]{fontenc}
\usepackage[russian]{babel}
\usepackage{cmap}
\usetheme{Boadilla}

\newcommand{\Ki}{\mathscr{K}}
\newcommand{\D}{\mathscr{D}}
\newcommand{\be}{\begin{equation}}
\newcommand{\ee}{\end{equation}}
\newcommand{\bes}{\begin{equation*}}
\newcommand{\ees}{\end{equation*}}

\makeatletter
\setbeamertemplate{footline}{%
  \leavevmode%
  \hbox{%
    \begin{beamercolorbox}[wd=.92\paperwidth,ht=2.25ex,dp=1ex,center]{title in head/foot}%
      \usebeamerfont{title in head/foot}\insertshorttitle
    \end{beamercolorbox}%
  }%
  \begin{beamercolorbox}[wd=.08\paperwidth,ht=2.25ex,dp=1ex,right]{date in head/foot}%
    \usebeamerfont{date in head/foot}%
    \usebeamertemplate{page number in head/foot}%
    \hspace*{2ex} 
  \end{beamercolorbox}
  \vskip0pt%
}
\makeatother

\title{\bf Урок 9. Тестирование гипотез (часть 1)}
\author{{\bf Хакимов Р.И. + ChatGPT}}
 \date[\today]{}

\begin{document}
\begin{frame}
\titlepage
\end{frame}
%------------------------------------------------

\begin{frame}
\frametitle{Введение в тестирование гипотез}
{\bf Тестирование гипотез} — это статистический метод, используемый для проверки предположений (гипотез) о параметрах генеральной совокупности на основе выборочных данных. Этот процесс помогает определить, насколько убедительными являются результаты исследования, и позволяет сделать выводы о популяции, исходя из данных выборки.
\end{frame}
%------------------------------------------------

\begin{frame}
\frametitle{Основные концепции тестирования гипотез}
{\bf Гипотеза:} Предположение о параметре генеральной совокупности, которое проверяется на основе данных выборки.\\  
{\bf Нулевая гипотеза ($H_0$):} Основное предположение, которое мы тестируем. Обычно это гипотеза о том, что нет эффекта или различий. Например, "средний доход мужчин и женщин одинаков".\\
{\bf Альтернативная гипотеза ($H_1$ или $H_a$):} Гипотеза, которая предполагает наличие эффекта или различий. Это гипотеза, которая рассматривается в случае, если нулевая гипотеза отвергнута. Например, "средний доход мужчин и женщин различен".
\end{frame}
%------------------------------------------------

\begin{frame}
\frametitle{Основные концепции тестирования гипотез}
{\bf Уровень значимости ($\alpha$):} Вероятность ошибки первого рода ($\alpha$), то есть вероятность отклонения нулевой гипотезы, когда она на самом деле верна. Обычно выбирается значение 0.05 или 0.01.\\
{\bf P-значение:} Вероятность получения результатов, которые столь же экстремальны, как и наблюдаемые, при условии, что нулевая гипотеза верна. Если P-значение меньше уровня значимости, нулевая гипотеза отвергается.
\end{frame}
%------------------------------------------------

\begin{frame}
\frametitle{Основные концепции тестирования гипотез}
{\bf Ошибка первого рода ($\alpha$):} Ошибка, при которой нулевая гипотеза отвергается, хотя она верна.\\
Ошибка первого рода важна для понимания, так как её последствия могут быть серьезными, и она учитывается при планировании экспериментов и интерпретации их результатов.
\newline\\
{\bf Ошибка второго рода ($\beta$):} Ошибка, при которой нулевая гипотеза не отвергается, хотя альтернативная гипотеза верна.\\
Ошибка второго рода особенно критична, когда важно не пропустить реальные эффекты или различия, например, в медицине, правосудии или научных исследованиях.
\end{frame}
%------------------------------------------------

\begin{frame}
\frametitle{Процесс тестирования гипотез}
{\bf 1. Формулировка гипотез:}\\
- Нулевая гипотеза ($H_0$): Предположение о том, что никакого эффекта или различия нет.\\
- Альтернативная гипотеза ($H_1$ или $H_a$): Альтернативное предположение, которое предполагает наличие эффекта или различия.\\
{\bf 2. Выбор уровня значимости ($\alpha$):}\\
- Обычно выбирается 0.05, что означает 5\% вероятность ошибки первого рода.\\
{\bf 3. Сбор и анализ данных:}\\
- Собираются данные и рассчитываются статистики теста (например, выборочное среднее, выборочная дисперсия).\\
{\bf 4. Расчет тестовой статистики:}\\
- Вычисляется статистика теста (например, t-статистика, Z-статистика) в зависимости от типа теста.
\end{frame}
%------------------------------------------------

\begin{frame}
\frametitle{Процесс тестирования гипотез}
{\bf 5. Определение P-значения:}\\
- {\it P-значение} - это вероятность получения результатов, которые столь же экстремальны, как и наблюдаемые, при условии, что нулевая гипотеза верна. Оно сравнивается с уровнем значимости.\\
- P-значение вычисляется на основе тестовой статистики и распределения.\\
{\bf 6. Принятие решения:}\\
- Если P-значение меньше уровня значимости ($\alpha$), нулевая гипотеза отвергается.\\
- Если P-значение больше уровня значимости, нет оснований для отклонения нулевой гипотезы.\\
{\bf 7. Интерпретация результатов:}\\
- Делается вывод о том, поддерживается ли нулевая гипотеза, или есть основания для её отклонения в пользу альтернативной гипотезы.
\end{frame}
%------------------------------------------------

\begin{frame}
\frametitle{Типы тестов гипотез}
{\bf Тесты для среднего значения:}\\
- Тест z: Используется, когда размер выборки большой и дисперсия генеральной совокупности известна.\\
- Тест t: Используется, когда размер выборки малый и дисперсия генеральной совокупности неизвестна.\\
{\bf Тесты для пропорций:}\\
- Тест для одной пропорции: Используется для проверки гипотез о пропорциях в одной выборке.\\
- Тест для разности пропорций: Используется для проверки гипотез о разнице пропорций между двумя выборками.\\
{\bf Тесты для дисперсий:}\\
- Тест $\chi^2$ (хи-квадрат): Используется для проверки гипотез о дисперсиях.\\
{\bf Тесты для независимости:}\\
- Тест $\chi^2$ на независимость: Проверяет зависимость между двумя категориальными переменными.
\end{frame}
%------------------------------------------------

\begin{frame}
\frametitle{Одновыборочный и двухвыборочный t-тест}
{\bf t-тест} — это статистический метод, используемый для проверки гипотез о среднем значении генеральной совокупности или для сравнения средних значений двух выборок.\\
t-тест особенно полезен, когда размер выборки небольшой и дисперсия генеральной совокупности неизвестна.\\
Выделяют два основных типа t-тестов: одновыборочный и двухвыборочный.
\end{frame}
%------------------------------------------------

\begin{frame}
\frametitle{Процедура одновыборочного $t$-теста}
{\bf Одновыборочный $t$-тест} используется для проверки гипотезы о среднем значении генеральной совокупности на основе одной выборки.\\
{\bf 1. Формулировка гипотез:}\\
- Нулевая гипотеза ($H_0$): \( \mu = \mu_0 \), где \( \mu \) — среднее значение генеральной совокупности, \( \mu_0 \) — предполагаемое среднее значение по нулевой гипотезе.\\
- Альтернативная гипотеза ($H_1$): \( \mu \ne \mu_0 \) (двусторонний тест), или \( \mu > \mu_0 \) (односторонний тест), или \( \mu < \mu_0 \) (односторонний тест).\\
{\bf 2. Расчет тестовой статистики:}\\
- Формула для $t$-статистики:
     \[
     t = \frac{\bar{X} - \mu_0}{S / \sqrt{n}}
     \]
где:
- \( \bar{X} \) — выборочное среднее,\\
- \( \mu_0 \) — значение среднего по нулевой гипотезе,\\
- \( S \) — выборочное стандартное отклонение,\\
- \( n \) — размер выборки.\\
\end{frame}
%------------------------------------------------

\begin{frame}
\frametitle{Процедура одновыборочного $t$-теста}
{\bf 3. Определение степени свободы:}
- Степени свободы ($df$) для одновыборочного $t$-теста: \( n - 1 \).\\
{\bf 4. Сравнение с критическим значением $t$:}\\
- Определите критическое значение $t$ из таблицы распределения $t$ на основе уровня значимости ($\alpha$) и степеней свободы.\\
- Если вычисленное $t$-значение больше критического (или если P-значение меньше $\alpha$), нулевая гипотеза отвергается.
\end{frame}
%------------------------------------------------

\begin{frame}
\frametitle{Процедура одновыборочного $t$-теста}
{\bf Пример.} Предположим, что вы хотите проверить, отличается ли среднее значение роста людей в вашей выборке от 170 см. Вы собрали выборку из 25 человек, средний рост составил 172 см, а выборочное стандартное отклонение — 8 см. Уровень значимости — 0.05.\\
- Нулевая гипотеза: \( \mu = 170 \).\\
- Альтернативная гипотеза: \( \mu \ne 170 \).\\
Расчет t-статистики:
\[
t = \frac{172 - 170}{8 / \sqrt{25}} = \frac{2}{1.6} = 1.25
\]
Если для 24 степеней свободы критическое значение t для двустороннего теста на уровне значимости 0.05 составляет 2.064, то поскольку 1.25 < 2.064, мы не отвергаем нулевую гипотезу.
\end{frame}
%------------------------------------------------

\begin{frame}
\frametitle{Процедура двухвыборочного $t$-теста}
{\bf Двухвыборочный $t$-тест} используется для сравнения средних значений двух независимых выборок, чтобы определить, есть ли статистически значимая разница между ними.\\
{\bf 1. Формулировка гипотез:}\\
{\it Нулевая гипотеза ($H_0$):} \( \mu_1 = \mu_2 \), где \( \mu_1 \) и \( \mu_2 \) — средние значения двух генеральных совокупностей.\\
{\it Альтернативная гипотеза ($H_1$):} \( \mu_1 \ne \mu_2 \) (двусторонний тест), или \( \mu_1 > \mu_2 \) (односторонний тест), или \( \mu_1 < \mu_2 \) (односторонний тест).\\
\end{frame}
%------------------------------------------------

\begin{frame}
\frametitle{Процедура двухвыборочного $t$-теста}
{\bf 2. Расчет тестовой статистики:}\\
- Формула для t-статистики:
     \[
     t = \frac{\bar{X}_1 - \bar{X}_2}{\sqrt{S_p^2 \left(\frac{1}{n_1} + \frac{1}{n_2}\right)}}
     \]
где:\\
- \( \bar{X}_1 \) и \( \bar{X}_2 \) — выборочные средние двух групп,\\
- \( S_p^2 \) — объединенная выборочная дисперсия,\\
- \( n_1 \) и \( n_2 \) — размеры выборок.\\
- Объединенная выборочная дисперсия рассчитывается как:
     \[
     S_p^2 = \frac{(n_1 - 1)S_1^2 + (n_2 - 1)S_2^2}{n_1 + n_2 - 2}
     \]
где \( S_1^2 \) и \( S_2^2 \) — выборочные дисперсии двух групп.
\end{frame}
%------------------------------------------------

\begin{frame}
\frametitle{Процедура двухвыборочного $t$-теста}
{\bf 3. Определение степени свободы:}\\
- Степени свободы ($df$) для двухвыборочного $t$-теста: \( n_1 + n_2 - 2 \).\\
{\bf 4. Сравнение с критическим значением $t$:}\\
- Определите критическое значение $t$ из таблицы распределения $t$ на основе уровня значимости ($\alpha$) и степеней свободы.\\
- Если вычисленное $t$-значение больше критического (или если P-значение меньше $\alpha$), нулевая гипотеза отвергается.
\end{frame}
%------------------------------------------------

\begin{frame}
\frametitle{Процедура двухвыборочного $t$-теста}
{\bf Пример.} Предположим, что вы хотите сравнить средний рост мужчин и женщин. Вы имеете две выборки: рост 20 мужчин (средний 175 см, стандартное отклонение 10 см) и рост 25 женщин (средний 165 см, стандартное отклонение 8 см). Уровень значимости — 0.05.\\
{\it Нулевая гипотеза:} \( \mu_1 = \mu_2 \).\\
{\it Альтернативная гипотеза:} \( \mu_1 \ne \mu_2 \).\\
Расчет объединенной дисперсии:
\[
S_p^2 = \frac{(20 - 1) \cdot 100 + (25 - 1) \cdot 64}{20 + 25 - 2} = \frac{1900 + 1536}{43} = 67.0
\]
Расчет t-статистики:
\[
t = \frac{175 - 165}{\sqrt{67.0 \left(\frac{1}{20} + \frac{1}{25}\right)}} = \frac{10}{\sqrt{67.0 \cdot 0.089}} = \frac{10}{1.76} = 5.68
\]
Если для 43 степеней свободы критическое значение t для двустороннего теста на уровне значимости 0.05 составляет 2.016, то поскольку 5.68 > 2.016, нулевая гипотеза отвергается.
\end{frame}
%------------------------------------------------

\begin{frame}
\frametitle{Одновыборочный и двухвыборочный t-тест}
{\bf Заключение}
\newline\\
t-тесты являются мощным инструментом для проверки гипотез о средних значениях и сравнении групп.\\
Одновыборочный t-тест помогает проверить, отличается ли среднее значение одной выборки от известного значения.\\
Двухвыборочный t-тест позволяет сравнить средние значения двух независимых выборок, чтобы выявить статистически значимые различия.
\end{frame}
%------------------------------------------------

\begin{frame}
\frametitle{Z-тест для сравнения средних значений}
{\bf z-тест для сравнения средних значений} используется для проверки гипотез о разнице между средними значениями двух независимых выборок.\\
Этот тест особенно полезен, когда размер выборок достаточно большой, и дисперсии генеральных совокупностей известны или можно считать их приближенными к известным значениям.
\end{frame}
%------------------------------------------------

\begin{frame}
\frametitle{Процедура z-теста для сравнения средних значений}
{\bf 1. Формулировка гипотез:}\\
{\it Нулевая гипотеза ($H_0$):} \( \mu_1 = \mu_2 \).  Средние значения двух генеральных совокупностей равны.\\
{\it Альтернативная гипотеза ($H_1$):} \( \mu_1 \ne \mu_2 \) (двусторонний тест). Средние значения двух генеральных совокупностей различны. Или \( \mu_1 > \mu_2 \) (односторонний тест), или \( \mu_1 < \mu_2 \) (односторонний тест).\\
{\bf 2. Выбор уровня значимости ($\alpha$):}\\
- Уровень значимости обычно устанавливается на уровне 0.05, 0.01 или другом значении, в зависимости от исследовательских целей.\\
{\bf 3. Сбор и анализ данных:}\\
- Собираются данные из двух независимых выборок.\\
- Вычисляются выборочные средние \( \bar{X}_1 \) и \( \bar{X}_2 \), а также стандартные отклонения \( \sigma_1 \) и \( \sigma_2 \) для каждой из выборок.
\end{frame}
%------------------------------------------------

\begin{frame}
\frametitle{Процедура z-теста для сравнения средних значений}
{\bf 4. Расчет тестовой статистики:}\\
{\it Формула для z-статистики:}
  \[
  z = \frac{(\bar{X}_1 - \bar{X}_2) - (\mu_1 - \mu_2)}{\sqrt{\frac{\sigma_1^2}{n_1} + \frac{\sigma_2^2}{n_2}}}
  \]
где:\\
- \( \bar{X}_1 \) и \( \bar{X}_2 \) — выборочные средние двух групп,\\
- \( \sigma_1^2 \) и \( \sigma_2^2 \) — дисперсии генеральных совокупностей (или выборочные дисперсии, если дисперсии генеральных совокупностей неизвестны),\\
- \( n_1 \) и \( n_2 \) — размеры выборок.\\
Если нулевая гипотеза предполагает равенство средних (\( \mu_1 - \mu_2 = 0 \)), формула упрощается до:
  \[
  z = \frac{\bar{X}_1 - \bar{X}_2}{\sqrt{\frac{\sigma_1^2}{n_1} + \frac{\sigma_2^2}{n_2}}}
  \]
\end{frame}
%------------------------------------------------

\begin{frame}
\frametitle{Процедура z-теста для сравнения средних значений}
{\bf 5. Определение критического значения и P-значения:}\\
- Определите критическое значение z из таблицы распределения нормальных случайных величин на основе уровня значимости ($\alpha$).\\
- Рассчитайте P-значение, которое соответствует вычисленному z-значению.\\
- Если P-значение меньше уровня значимости ($\alpha$), отвергайте нулевую гипотезу.\\
{\bf 6. Принятие решения:}\\
- Отклонение нулевой гипотезы: Если вычисленное z-значение больше критического значения (или если P-значение меньше $\alpha$), нулевая гипотеза отвергается.\\
- Не отклонение нулевой гипотезы: Если вычисленное z-значение меньше критического значения (или если P-значение больше $\alpha$), нет оснований для отклонения нулевой гипотезы.
\end{frame}
%------------------------------------------------

\begin{frame}
\frametitle{Процедура z-теста для сравнения средних значений}
{\bf Пример.} Предположим, вы хотите сравнить средний вес мужчин и женщин в двух независимых группах. В выборке из 30 мужчин средний вес составляет 80 кг, стандартное отклонение — 10 кг. В выборке из 40 женщин средний вес составляет 70 кг, стандартное отклонение — 8 кг. Уровень значимости — 0.05.\\
- Нулевая гипотеза: \( \mu_1 = \mu_2 \).\\
- Альтернативная гипотеза: \( \mu_1 \ne \mu_2 \).\\
Расчет z-статистики:
\[
z = \frac{80 - 70}{\sqrt{\frac{10^2}{30} + \frac{8^2}{40}}} = \frac{10}{\sqrt{3.33 + 1.60}} = \frac{10}{2.50} = 4.00
\]
Для уровня значимости 0.05 и двустороннего теста критическое значение z приблизительно равно ±1.96. Поскольку 4.00 > 1.96, нулевая гипотеза отвергается, что означает, что средние веса мужчин и женщин статистически значимо различаются.
\end{frame}
%------------------------------------------------

\begin{frame}
\frametitle{Z-тест для сравнения средних значений}
{\bf Заключение}
\newline\\
z-тест для сравнения средних значений полезен при анализе разницы между двумя независимыми группами, когда размеры выборок достаточно велики и дисперсии генеральных совокупностей известны.\\
Этот тест позволяет определить, есть ли статистически значимая разница между средними значениями двух групп.
\end{frame}

\end{document}