\documentclass[aspectratio=169]{beamer}
%\documentclass{beamer}
\usepackage[utf8]{inputenc}
\usepackage[T2A]{fontenc}
\usepackage[russian]{babel}
\usepackage{cmap}
\usetheme{Boadilla}

\newcommand{\Ki}{\mathscr{K}}
\newcommand{\D}{\mathscr{D}}
\newcommand{\be}{\begin{equation}}
\newcommand{\ee}{\end{equation}}
\newcommand{\bes}{\begin{equation*}}
\newcommand{\ees}{\end{equation*}}

\makeatletter
\setbeamertemplate{footline}{%
  \leavevmode%
  \hbox{%
    \begin{beamercolorbox}[wd=.92\paperwidth,ht=2.25ex,dp=1ex,center]{title in head/foot}%
      \usebeamerfont{title in head/foot}\insertshorttitle
    \end{beamercolorbox}%
  }%
  \begin{beamercolorbox}[wd=.08\paperwidth,ht=2.25ex,dp=1ex,right]{date in head/foot}%
    \usebeamerfont{date in head/foot}%
    \usebeamertemplate{page number in head/foot}%
    \hspace*{2ex} 
  \end{beamercolorbox}
  \vskip0pt%
}
\makeatother

\title{\bf Урок 10. Тестирование гипотез (часть 2)}
\author{{\bf Хакимов Р.И. + ChatGPT}}
 \date[\today]{}

\begin{document}
\begin{frame}
\titlepage
\end{frame}
%------------------------------------------------

\begin{frame}
\frametitle{ANOVA: анализ дисперсии между группами}
{\bf ANOVA (анализ дисперсии)} — это статистический метод, используемый для проверки наличия статистически значимых различий между средними значениями более чем двух групп.\\
ANOVA позволяет определить, существует ли значительная разница между группами, не требуя проверки всех возможных парных сравнений.
\end{frame}
%------------------------------------------------

\begin{frame}
\frametitle{ANOVA: анализ дисперсии между группами}
{\bf Основные концепции ANOVA}
\newline\\
{\bf Цель ANOVA:}\\
- Оценить, существует ли статистически значимая разница между средними значениями нескольких групп.\\
- Определить, какую долю общей дисперсии в данных можно объяснить различиями между группами, а какую — случайными колебаниями внутри групп.
\newline\\
{\bf Типы ANOVA:}\\
{\it - Однофакторный ANOVA:} используется для проверки разницы между средними значениями более чем двух групп по одному фактору (например, различия в тестовых оценках студентов, обучавшихся по разным методикам).\\
{\it - Многофакторный ANOVA:} используется для анализа влияния двух или более факторов на средние значения (например, влияние методики обучения и пола на тестовые оценки).
\end{frame}
%------------------------------------------------

\begin{frame}
\frametitle{Процедура однофакторного ANOVA}
{\bf 1. Формулировка гипотез:}\\
{\it Нулевая гипотеза ($H_0$):} Все средние значения групп равны. Формально: \( \mu_1 = \mu_2 = \cdots = \mu_k \), где \( k \) — количество групп.\\
{\it Альтернативная гипотеза ($H_1$):} По крайней мере одно из средних значений групп отличается от остальных.\\
{\bf 2. Сбор и анализ данных:}\\
- Собираются данные из нескольких независимых групп.\\
- Вычисляются выборочные средние и дисперсии для каждой группы.\\
\end{frame}
%------------------------------------------------

\begin{frame}
\frametitle{Процедура однофакторного ANOVA}
{\bf 3. Расчет дисперсий:}\\
{\it Общая дисперсия (SST):} Общая вариация всех наблюдений относительно общего среднего.
  \[
  SST = \sum_{i=1}^{k} \sum_{j=1}^{n_i} (X_{ij} - \bar{X})^2
  \]
где \( X_{ij} \) — значение j-го наблюдения в i-й группе, \( \bar{X} \) — общее среднее значение.\\
{\it Дисперсия между группами (SSB):} Вариация, объясненная различиями между группами.
  \[
  SSB = \sum_{i=1}^{k} n_i (\bar{X}_i - \bar{X})^2
  \]
где \( \bar{X}_i \) — среднее значение i-й группы, \( n_i \) — размер i-й группы.
\end{frame}
%------------------------------------------------

\begin{frame}
\frametitle{Процедура однофакторного ANOVA}
{\bf 3. Расчет дисперсий:}\\
{\it Дисперсия внутри групп (SSW):} Вариация внутри групп.
  \[
  SSW = \sum_{i=1}^{k} \sum_{j=1}^{n_i} (X_{ij} - \bar{X}_i)^2
  \]
\end{frame}
%------------------------------------------------

\begin{frame}
\frametitle{Процедура однофакторного ANOVA}
{\bf 4. Расчет средних квадратов и F-статистики:}\\
{\it Средний квадрат между группами (MSB):}\\
  \[
  MSB = \frac{SSB}{k - 1}
  \]
где \( k \) — количество групп.\\
{\it Средний квадрат внутри групп (MSW):}\\
  \[
  MSW = \frac{SSW}{N - k}
  \]
где \( N \) — общее количество наблюдений.\\
{\it F-статистика:}
  \[
  F = \frac{MSB}{MSW}
  \]
Значение F-статистики используется для проверки гипотез.
\end{frame}
%------------------------------------------------

\begin{frame}
\frametitle{Процедура однофакторного ANOVA}
{\bf 5. Определение критического значения и P-значения:}\\
- Определите критическое значение F из таблицы распределения F на основе уровня значимости ($\alpha$), степеней свободы между группами \( (k - 1) \) и степеней свободы внутри групп \( (N - k) \).\\
- Рассчитайте P-значение, которое соответствует вычисленному F-значению.\\
- Если P-значение меньше уровня значимости ($\alpha$), отвергайте нулевую гипотезу.\\
{\bf 6. Принятие решения:}\\
{\it Отклонение нулевой гипотезы:} Если вычисленное F-значение больше критического значения (или если P-значение меньше $\alpha$), нулевая гипотеза отвергается.\\
{\it Не отклонение нулевой гипотезы:} Если F-значение меньше критического значения (или если P-значение больше $\alpha$), нет оснований отвергать нулевую гипотезу.
\end{frame}
%------------------------------------------------

\begin{frame}
\frametitle{Пример однофакторного ANOVA}
Предположим, вы хотите сравнить средние оценки по тесту у студентов, обучавшихся по трем различным методикам. Вы собрали данные из трех групп студентов, каждая из которых обучалась по одной из методик:\\
{\it Группа 1:} 30 студентов, средняя оценка 75, стандартное отклонение 10.\\
{\it Группа 2:} 35 студентов, средняя оценка 80, стандартное отклонение 12.\\
{\it Группа 3:} 25 студентов, средняя оценка 70, стандартное отклонение 8.\\
{\it Уровень значимости:} 0.05.
\end{frame}
%------------------------------------------------

\begin{frame}
\frametitle{Пример однофакторного ANOVA}
{\bf 1. Формулировка гипотез:}\\
{\it $H_0$:} Средние оценки у всех групп равны.\\
{\it $H_1$:} По крайней мере одна средняя оценка отличается.\\
{\bf 2. Расчет дисперсий:}\\
- Общая дисперсия (SST) и дисперсия между группами (SSB) рассчитываются на основе всех наблюдений.\\
{\bf 3. Расчет средних квадратов и F-статистики:}\\
- Средний квадрат между группами (MSB) и средний квадрат внутри групп (MSW) рассчитываются на основе дисперсий.\\
- Примерный расчет F-статистики:
     \[
     F = \frac{MSB}{MSW}
     \]
\end{frame}
%------------------------------------------------

\begin{frame}
\frametitle{Пример однофакторного ANOVA}
{\bf 4. Определение критического значения и P-значения:}\\
- Используя таблицу распределения F для 2 степеней свободы между группами и 87 степеней свободы внутри групп, определите критическое значение и P-значение.\\
{\bf 5. Принятие решения:}\\
- Сравните F-значение с критическим значением и интерпретируйте P-значение.
\end{frame}
%------------------------------------------------

\begin{frame}
\frametitle{ANOVA: анализ дисперсии между группами}
{\bf Заключение}\\
ANOVA — мощный инструмент для анализа различий между средними значениями нескольких групп.\\
Однофакторный ANOVA позволяет проверить, существуют ли значимые различия между группами по одному фактору. Этот метод помогает избежать множественных сравнений и предоставляет общее представление о влиянии фактора на результаты.
\end{frame}
%------------------------------------------------

\begin{frame}
\frametitle{Практические задания по тестированию гипотез}
Практические задания по тестированию гипотез помогают закрепить теоретические знания и навыки, применяя их в реальных или приближенных к реальности ситуациях.\\
Вот несколько примеров практических заданий по тестированию гипотез, которые включают различные методы:
\end{frame}
%------------------------------------------------

\begin{frame}
\frametitle{Одновыборочный t-тест. Практика}
{\bf Задание:}\\
В компании проводят опрос среди сотрудников о среднем уровне стресса.\\
Средний балл по опросу для выборки из 50 сотрудников составляет 75, выборочное стандартное отклонение — 10.\\
Известно, что средний уровень стресса в прошлом году был 70.\\
Проверьте, изменился ли средний уровень стресса в этом году с помощью одновыборочного t-теста.\\
Уровень значимости — 0.05.
\end{frame}
%------------------------------------------------

\begin{frame}
\frametitle{Одновыборочный t-тест. Практика}
{\bf Решение:}\\
{\bf 1. Формулировка гипотез:}\\
- $H_0$: \( \mu = 70 \)\\
- $H_1$: \( \mu \ne 70 \)\\
{\bf 2. Расчет t-статистики:}\\
   \[
   t = \frac{75 - 70}{10 / \sqrt{50}} = \frac{5}{1.414} = 3.54
   \]
{\bf 3. Определение критического значения:}\\
Для 49 степеней свободы и уровня значимости 0.05 (двусторонний тест), критическое значение t приблизительно равно ±2.009.\\
{\bf 4. Сравнение t-значения с критическим:}\\
3.54 > 2.009, поэтому нулевая гипотеза отвергается.
\newline\\
{\bf Вывод:} Средний уровень стресса изменился по сравнению с прошлым годом.
\end{frame}
%------------------------------------------------

\begin{frame}
\frametitle{Двухвыборочный t-тест. Практика}
{\bf Задание:}\\
Вы хотите сравнить средний уровень удовлетворенности клиентов двух разных продуктов.\\
У вас есть следующие данные:\\
- Продукт A: среднее удовлетворение 80, выборочное стандартное отклонение 12, размер выборки 30.\\
- Продукт B: среднее удовлетворение 75, выборочное стандартное отклонение 10, размер выборки 35.\\
Проверьте, есть ли статистически значимая разница в уровне удовлетворенности между двумя продуктами.\\
Уровень значимости — 0.05.
\end{frame}
%------------------------------------------------

\begin{frame}
\frametitle{Двухвыборочный t-тест. Практика}
{\bf Решение:}\\
{\bf 1. Формулировка гипотез:}\\
- $H_0$: \( \mu_A = \mu_B \)\\
- $H_1$: \( \mu_A \ne \mu_B \)\\
{\bf 2. Расчет дисперсий и t-статистики:}\\
{\it Объединенная дисперсия:}
     \[
     S_p^2 = \frac{(30 - 1) \cdot 12^2 + (35 - 1) \cdot 10^2}{30 + 35 - 2} = \frac{29 \cdot 144 + 34 \cdot 100}{63} = 121.11
     \]
{\it Средний квадрат между группами:}
     \[
     t = \frac{80 - 75}{\sqrt{121.11 \left(\frac{1}{30} + \frac{1}{35}\right)}} = \frac{5}{\sqrt{121.11 \cdot 0.0714}} = \frac{5}{2.60} = 1.92
     \]
\end{frame}
%------------------------------------------------

\begin{frame}
\frametitle{Двухвыборочный t-тест. Практика}
{\bf Решение:}\\
{\bf 3. Определение критического значения:}\\
Для 63 степеней свободы и уровня значимости 0.05 (двусторонний тест), критическое значение t приблизительно равно ±2.000.\\
{\bf 4. Сравнение t-значения с критическим:}\\
1.92 < 2.000, поэтому нулевая гипотеза не отвергается.
\newline\\
{\bf Вывод:} Нет статистически значимой разницы в уровне удовлетворенности между продуктами A и B.
\end{frame}
%------------------------------------------------

\begin{frame}
\frametitle{ANOVA (Однофакторный анализ дисперсии). Практика}
{\bf Задание:}\\
Исследователи хотят сравнить средние баллы по экзаменам у студентов, обучавшихся по трем различным методикам.\\
Получены следующие данные:\\
- Методика 1: 20 студентов, средний балл 78, стандартное отклонение 8.\\
- Методика 2: 25 студентов, средний балл 85, стандартное отклонение 10.\\
- Методика 3: 30 студентов, средний балл 80, стандартное отклонение 9.\\
Проверьте, есть ли статистически значимые различия между средними баллами студентов, обучавшихся по разным методикам.\\
Уровень значимости — 0.05.
\end{frame}
%------------------------------------------------

\begin{frame}
\frametitle{ANOVA (Однофакторный анализ дисперсии). Практика}
{\bf Решение:}\\
{\bf 1. Формулировка гипотез:}\\
- $H_0$: Все средние баллы равны.\\
- $H_1$: По крайней мере один средний балл отличается.\\
{\bf 2. Расчет дисперсий:}\\
- Общая дисперсия (SST), дисперсия между группами (SSB) и дисперсия внутри групп (SSW) рассчитываются на основе предоставленных данных.\\
{\bf 3. Расчет средних квадратов и F-статистики:}\\
- Средний квадрат между группами (MSB) и средний квадрат внутри групп (MSW) рассчитываются.\\
- Формула для F-статистики:
     \[
     F = \frac{MSB}{MSW}
     \]
\end{frame}
%------------------------------------------------

\begin{frame}
\frametitle{ANOVA (Однофакторный анализ дисперсии). Практика}
{\bf Решение:}\\
{\bf 4. Определение критического значения и P-значения:}\\
- Определите критическое значение F из таблицы распределения F для соответствующих степеней свободы и уровня значимости.\\
{\bf 5. Сравнение F-значения с критическим:}\\
Если F-значение больше критического значения, нулевая гипотеза отвергается.
\newline\\
{\bf Вывод:}\\
На основе вычислений можно сделать вывод о наличии или отсутствии статистически значимых различий между средними баллами по различным методикам.
\end{frame}
%------------------------------------------------

\begin{frame}
\frametitle{z-тест для сравнения средних значений. Практика}
{\bf Задание:}\\
В двух независимых выборках были измерены результаты тестов:\\
- Выборка 1: средний результат 90, стандартное отклонение 15, размер выборки 50.\\
- Выборка 2: средний результат 85, стандартное отклонение 20, размер выборки 60.\\
Проверьте, существует ли статистически значимая разница между средними результатами тестов для двух выборок.\\
Уровень значимости — 0.01.
\end{frame}
%------------------------------------------------

\begin{frame}
\frametitle{z-тест для сравнения средних значений. Практика}
{\bf Решение:}\\
{\bf 1. Формулировка гипотез:}\\
- $H_0$: \( \mu_1 = \mu_2 \)\\
- $H_1$: \( \mu_1 \ne \mu_2 \)\\
{\bf 2. Расчет z-статистики:}\\
   \[
   z = \frac{(90 - 85)}{\sqrt{\frac{15^2}{50} + \frac{20^2}{60}}} = \frac{5}{\sqrt{4.50 + 6.67}} = \frac{5}{2.79} = 1.79
   \]
{\bf 3. Определение критического значения:}\\
Для уровня значимости 0.01 и двустороннего теста, критическое значение z приблизительно равно ±2.576.\\
{\bf 4. Сравнение z-значения с критическим:}\\
1.79 < 2.576, поэтому нулевая гипотеза не отвергается.\\
{\bf Вывод:} Нет статистически значимой разницы между средними результатами тестов двух выборок.
\end{frame}

\end{document}