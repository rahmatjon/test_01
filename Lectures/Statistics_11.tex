\documentclass[aspectratio=169]{beamer}
%\documentclass{beamer}
\usepackage[utf8]{inputenc}
\usepackage[T2A]{fontenc}
\usepackage[russian]{babel}
\usepackage{cmap}
\usetheme{Boadilla}

\newcommand{\Ki}{\mathscr{K}}
\newcommand{\D}{\mathscr{D}}
\newcommand{\be}{\begin{equation}}
\newcommand{\ee}{\end{equation}}
\newcommand{\bes}{\begin{equation*}}
\newcommand{\ees}{\end{equation*}}

\makeatletter
\setbeamertemplate{footline}{%
  \leavevmode%
  \hbox{%
    \begin{beamercolorbox}[wd=.92\paperwidth,ht=2.25ex,dp=1ex,center]{title in head/foot}%
      \usebeamerfont{title in head/foot}\insertshorttitle
    \end{beamercolorbox}%
  }%
  \begin{beamercolorbox}[wd=.08\paperwidth,ht=2.25ex,dp=1ex,right]{date in head/foot}%
    \usebeamerfont{date in head/foot}%
    \usebeamertemplate{page number in head/foot}%
    \hspace*{2ex} 
  \end{beamercolorbox}
  \vskip0pt%
}
\makeatother

\title{\bf Урок 11. Корреляционный анализ}
\author{{\bf Хакимов Р.И. + ChatGPT}}
 \date[\today]{}

\begin{document}
\begin{frame}
\titlepage
\end{frame}
%------------------------------------------------

\begin{frame}
\frametitle{Введение в корреляцию}
{\bf Корреляция} -- это статистический метод, используемый для измерения и анализа степени и направления взаимосвязи между двумя количественными переменными.
\newline\\
Основная цель корреляционного анализа -- определить, насколько сильно и в каком направлении одна переменная связана с другой.
\newline\\
Корреляция используется в различных областях, таких как экономика, медицина, социология и психология, для изучения взаимосвязей между переменными и для построения предсказательных моделей.
\end{frame}
%------------------------------------------------

\begin{frame}
\frametitle{Основные концепции корреляции}
{\bf Корреляционный коэффициент}\\
Это мера силы и направления линейной зависимости между двумя переменными.\\
Значения корреляционного коэффициента (r) варьируются от -1 до 1:\\
\quad\( r = 1 \): Идеальная положительная линейная зависимость.\\
\quad\( r = -1 \): Идеальная отрицательная линейная зависимость.\\
\quad\( r = 0 \): Отсутствие линейной зависимости.\\
{\bf Типы корреляции:}\\
{\it Положительная корреляция:} Если одна переменная увеличивается, то другая переменная также увеличивается.\\
{\it Отрицательная корреляция:} Если одна переменная увеличивается, то другая переменная уменьшается.\\
{\bf Корреляция и линейная зависимость:}\\
Корреляция измеряет только линейную зависимость между переменными. Если зависимость нелинейная, корреляция может не отражать истинную связь.
\end{frame}
%------------------------------------------------

\begin{frame}
\frametitle{Коэффициент корреляции Пирсона}
{\bf Коэффициент корреляции Пирсона} измеряет линейную зависимость между двумя переменными. Он может принимать значения от -1 (идеальная обратная корреляция) до 1 (идеальная прямая корреляция), а 0 указывает на отсутствие линейной зависимости.\\
{\bf Формула для вычисления коэффициента корреляции Пирсона $(r)$:}
     \[
     r = \frac{\sum_{i=1}^{n} (X_i - \bar{X})(Y_i - \bar{Y})}{\sqrt{\sum_{i=1}^{n} (X_i - \bar{X})^2 \sum_{i=1}^{n} (Y_i - \bar{Y})^2}}
     \]
где \(X_i\) и \(Y_i\) — значения двух переменных, \(\bar{X}\) и \(\bar{Y}\) — средние значения переменных.\\
Этот коэффициент применяется, когда данные нормально распределены и связь между переменными линейная.
\end{frame}
%------------------------------------------------

\begin{frame}
\frametitle{Коэффициент корреляции Пирсона}
{\bf Шаги вычисления:}\\
1. Найдите средние значения \( \bar{x} \) и \( \bar{y} \) для каждой переменной \( x \) и \( y \):
   \[
   \bar{x} = \frac{\sum x_i}{n}, \quad \bar{y} = \frac{\sum y_i}{n}
   \]
2. Вычислите отклонения каждого значения от среднего для обеих переменных: \( (x_i - \bar{x}) \) и \( (y_i - \bar{y}) \).\\
3. Найдите произведение отклонений для каждой пары значений \( (x_i - \bar{x})(y_i - \bar{y}) \).\\
4. Просуммируйте произведения отклонений:\\
   \[
   \sum (x_i - \bar{x})(y_i - \bar{y})
   \]
5. Найдите квадрат отклонений для каждой переменной: \( (x_i - \bar{x})^2 \) и \( (y_i - \bar{y})^2 \).\\
\end{frame}
%------------------------------------------------

\begin{frame}
\frametitle{Коэффициент корреляции Пирсона}
{\bf Шаги вычисления:}\\
6. Просуммируйте квадраты отклонений для обеих переменных:
   \[
   \sum (x_i - \bar{x})^2 \quad \text{и} \quad \sum (y_i - \bar{y})^2
   \]
7. Подставьте значения в формулу для расчёта коэффициента Пирсона:
   \[
   r = \frac{\sum (x_i - \bar{x})(y_i - \bar{y})}{\sqrt{\sum (x_i - \bar{x})^2 \sum (y_i - \bar{y})^2}}
   \]
{\bf Интерпретация:}\\
\quad - \( r = 1 \) — идеальная положительная корреляция,\\
\quad - \( r = -1 \) — идеальная отрицательная корреляция,\\
\quad - \( r = 0 \) — отсутствует линейная зависимость.\\
\end{frame}
%------------------------------------------------

\begin{frame}
\frametitle{Коэффициент корреляции Пирсона. Пример}
Предположим, вы хотите исследовать зависимость между количеством часов, проведенных за изучением, и оценками на экзамене. Вы собрали данные для группы студентов:\\
\begin{center}
\begin{tabular}{ |c|c| } 
 \hline
 Часы изучения & Оценка на экзамене\\ 
 \hline
 1 & 55\\
 \hline
 2 & 60\\
 \hline
 3 & 65\\
 \hline
 4 & 70\\
 \hline
 5 & 75\\
 \hline
\end{tabular}
\end{center}
Для вычисления коэффициента корреляции Пирсона:\\
1. Вычислите средние значения для двух переменных:
   \[
   \bar{X} = \frac{1 + 2 + 3 + 4 + 5}{5} = 3
   \]
   \[
   \bar{Y} = \frac{55 + 60 + 65 + 70 + 75}{5} = 65
   \]
\end{frame}
%------------------------------------------------

\begin{frame}
\frametitle{Коэффициент корреляции Пирсона. Пример}
2. Вычислите сумму произведений отклонений от средних:
   \[
   \sum (X_i - \bar{X})(Y_i - \bar{Y})
   \]
   Для каждой пары (X, Y):\\
   \quad (1 - 3)(55 - 65) = 20\\
   \quad (2 - 3)(60 - 65) = 5\\
   \quad (3 - 3)(65 - 65) = 0\\
   \quad (4 - 3)(70 - 65) = 5\\
   \quad (5 - 3)(75 - 65) = 20\\
   \quad Сумма = 50\\
3. Вычислите сумму квадратов отклонений:\\
Сумма квадратов отклонений X: $\sum (X_i - \bar{X})^2 = 4 + 1 + 0 + 1 + 4 = 10.$\\
Сумма квадратов отклонений Y: $\sum (Y_i - \bar{Y})^2 = 100 + 25 + 0 + 25 + 100 = 250$\\
\end{frame}
%------------------------------------------------

\begin{frame}
\frametitle{Коэффициент корреляции Пирсона. Пример}
4. Расчет коэффициента корреляции:
   \[
   r = \frac{50}{\sqrt{10 \times 250}} = \frac{50}{\sqrt{2500}} = \frac{50}{50} = 1
   \]
{\bf Вывод:} Коэффициент корреляции \( r = 1 \) указывает на идеальную положительную линейную зависимость между часами изучения и оценками на экзамене.
\end{frame}
%------------------------------------------------

\begin{frame}
\frametitle{Коэффициент корреляции Спирмена}
{\bf Коэффициент корреляции Спирмена} -- это непараметрическая мера статистической зависимости между двумя переменными. Он основан на рангах значений, а не на самих значениях.\\
Используется для измерения силы и направления монотонной (не обязательно линейной) зависимости между переменными. Полезен, когда данные не соответствуют нормальному распределению.\\
{\bf Формула для вычисления коэффициента корреляции Спирмена:}\\
\[
r_s = 1 - \frac{6 \sum d_i^2}{n(n^2 - 1)}
\]
где:\\
- \( r_s \) — коэффициент корреляции Спирмена,\\
- \( d_i \) — разница между рангами соответствующих значений двух переменных для каждой пары наблюдений,\\
- \( n \) — количество наблюдений.
\end{frame}
%------------------------------------------------

\begin{frame}
\frametitle{Коэффициент корреляции Спирмена}
{\bf Шаги вычисления:}\\
1. Упорядочьте значения каждой переменной в виде рангов.\\
2. Найдите разницу рангов для каждой пары значений.\\
3. Возведите разницу рангов \( d_i \) в квадрат.\\
4. Просуммируйте квадраты разностей рангов.\\
5. Подставьте полученные значения в формулу.
\newline\\
{\bf Коэффициент \( r_s \) принимает значения от -1 до 1:}\\
\quad \( r_s = 1 \) указывает на полную положительную зависимость (идеальная прямая связь),\\
\quad \( r_s = -1 \) указывает на полную отрицательную зависимость (идеальная обратная связь),\\
\quad \( r_s = 0 \) говорит об отсутствии корреляции.
\end{frame}
%------------------------------------------------

\begin{frame}
\frametitle{Коэффициент корреляции Спирмена. Пример}
Рассмотрим следующий набор данных, в котором представлены оценки студентов по математике и физике:
\begin{center}
\begin{tabular}{ |c|c|c| } 
 \hline
Студент & Математика (X) & Физика (Y)\\ 
 \hline
 1 & 85 & 80\\
 \hline
 2 & 90 & 85\\
 \hline
 3 & 78 & 70\\
 \hline
 4 & 92 & 90\\
 \hline
 5 & 88 & 82\\
 \hline
\end{tabular}
\end{center}
\end{frame}
%------------------------------------------------

\begin{frame}
\frametitle{Коэффициент корреляции Спирмена. Пример}
{\bf Шаг 1: Присвоение рангов}\\
Сначала мы присваиваем ранги значениям в обеих колонках. Если два значения одинаковы, им присваивается средний ранг:
\begin{center}
\begin{tabular}{ |c|c|c|c|c| } 
 \hline
Студент & Математика (X) & Ранг X & Физика (Y) & Ранг Y\\ 
 \hline
 1 & 85 & 2  & 83 & 3\\
 \hline
 2 & 90 & 4 & 85 & 4\\
 \hline
 3 & 72 & 1 & 75 & 1\\
 \hline
 4 & 92 & 5 & 90 & 5\\
 \hline
 5 & 88 & 3 & 81 & 2\\
 \hline
\end{tabular}
\end{center}
\end{frame}
%------------------------------------------------

\begin{frame}
\frametitle{Коэффициент корреляции Спирмена. Пример}
{\bf Шаг 2: Вычисление разностей рангов}\\
Теперь вычисляем разности рангов \( d_i \):
\begin{center}
\begin{tabular}{ |c|c|c|c|c| } 
 \hline
Студент & Ранг X & Ранг Y & \( d_i = \text{Ранг X} - \text{Ранг Y} \) & \( d_i^2 \)\\ 
 \hline
 1 & 2  & 3 & -1  & 1\\
 \hline
 2 & 4 & 4 & 0 & 0\\
 \hline
 3 & 1 & 1 & 0 & 0\\
 \hline
 4 & 5 & 5 & 0 & 0\\
 \hline
 5 & 3 & 2 & 1 & 1\\
 \hline
\end{tabular}
\end{center}

{\bf Шаг 3: Подсчет суммы квадратов разностей}\\
Теперь подсчитываем сумму квадратов разностей:
\[
\sum d_i^2 = 1 + 0 + 0 + 0 + 1 = 2
\]
\end{frame}
%------------------------------------------------

\begin{frame}
\frametitle{Коэффициент корреляции Спирмена. Пример}
{\bf Шаг 4: Подстановка в формулу}\\
Теперь можем подставить значения в формулу для вычисления коэффициента корреляции Спирмена:
\[
\rho_s = 1 - \frac{6 \times 2}{5(5^2 - 1)} = 1 - \frac{12}{5 \times 24} = 1 - \frac{12}{120} = 1 - 0.1 = 0.9
\]
{\bf Интерпретация результата}\\
Коэффициент корреляции Спирмена равен \( 0.9 \), что указывает на положительную монотонную зависимость между оценками студентов по математике и физике.
\end{frame}
%------------------------------------------------

\begin{frame}
\frametitle{Графическое представление корреляции}
{\bf Графическое представление корреляции} позволяет наглядно увидеть взаимосвязь между двумя количественными переменными. Это помогает в интерпретации данных и понимании природы их связи. Основные методы графического представления корреляции включают диаграмму рассеяния и линии тренда.
\end{frame}
%------------------------------------------------

\begin{frame}
\frametitle{Графическое представление корреляции}
{\bf Диаграмма рассеяния (scatter plot)} — это графическое представление данных, где каждая точка на плоскости соответствует одной паре значений переменных. Она позволяет увидеть, как значения одной переменной соотносятся со значениями другой переменной и выявить наличие линейной или нелинейной зависимости.
\newline\\
{\bf Структура диаграммы рассеяния:}\\
\quad - Ось X: значения первой переменной (например, количество часов изучения).\\
\quad - Ось Y: значения второй переменной (например, оценка на экзамене).
\end{frame}
%------------------------------------------------

\begin{frame}
\frametitle{Графическое представление корреляции}
{\bf Как интерпретировать диаграмму рассеяния:}\\
{\it 1. Положительная линейная зависимость:} Если точки диаграммы располагаются вдоль восходящей линии, это указывает на положительную корреляцию. Например, увеличение часов изучения связано с увеличением оценок на экзамене.\\
{\it 2. Отрицательная линейная зависимость:} Если точки диаграммы располагаются вдоль нисходящей линии, это указывает на отрицательную корреляцию. Например, увеличение температуры связано с уменьшением времени, проведенного на улице.\\
{\it 3. Отсутствие явной зависимости:} Если точки распределены случайным образом, это может указывать на отсутствие или слабую корреляцию между переменными.\\
{\it 4. Нелинейные зависимости:} Если точки формируют кривую, это может указывать на нелинейную зависимость между переменными.
\end{frame}
%------------------------------------------------

\begin{frame}
\frametitle{Графическое представление корреляции}
{\bf Линия тренда (или линия регрессии)} добавляется к диаграмме рассеяния, чтобы лучше визуализировать линейную зависимость между переменными. Она показывает направление и силу линейной связи между переменными.\\
{\bf Как построить линию тренда:}\\
1. Выполните линейную регрессию для определения линии тренда.\\
2. Постройте линию на диаграмме, используя уравнение регрессии. Обычно эта линия имеет вид \( Y = a + bX \), где \( a \) — свободный член, а \( b \) — коэффициент наклона (угол наклона).\\
{\bf Интерпретация:}\\
- Если линия тренда имеет положительный наклон, это указывает на положительную корреляцию: больше часов тренировки связано с лучшими результатами в соревнованиях.\\
- Если линия тренда почти горизонтальна, это указывает на отсутствие значительной корреляции.
\end{frame}
%------------------------------------------------

\begin{frame}
\frametitle{Интерпретация корреляционного анализа}
{\bf Интерпретация корреляционного анализа} включает в себя понимание и объяснение результатов, полученных при вычислении коэффициента корреляции, а также выводы, которые можно сделать на основе графического представления данных.\\
Далее некоторые ключевые аспекты интерпретации корреляционного анализа...
\end{frame}
%------------------------------------------------

\begin{frame}
\frametitle{Интерпретация корреляционного анализа}
{\bf 1. Коэффициент корреляции}\\
{\it Коэффициент корреляции (r)} измеряет силу и направление линейной зависимости между двумя количественными переменными. Он может принимать значения от -1 до 1:\\
\quad \( r = 1 \): Идеальная положительная линейная зависимость. Все точки на диаграмме рассеяния лежат на прямой линии, направленной вверх.\\
\quad \( r = -1 \): Идеальная отрицательная линейная зависимость. Все точки на диаграмме рассеяния лежат на прямой линии, направленной вниз.\\
\quad \( r = 0 \): Отсутствие линейной зависимости. Точки на диаграмме рассеяния распределены случайным образом.\\
{\it Сила корреляции:}\\
\quad $0 < |r| < 0.3$: Слабая корреляция.\\
\quad $0.3 \leq |r| < 0.7$: Умеренная корреляция.\\
\quad $0.7 \leq |r| \leq 1$: Сильная корреляция.
\end{frame}
%------------------------------------------------

\begin{frame}
\frametitle{Интерпретация корреляционного анализа}
{\bf 1. Коэффициент корреляции}\\

{\it Направление корреляции:}\\
\quad $r > 0$: Положительная корреляция. Когда одна переменная увеличивается, другая также увеличивается.\\
\quad $r < 0$: Отрицательная корреляция. Когда одна переменная увеличивается, другая уменьшается.\\
{\bf Пример интерпретации:}\\
Если коэффициент корреляции между количеством часов работы и производительностью составляет $0.85$, это указывает на сильную положительную корреляцию. Это может значить, что увеличение количества часов работы связано с увеличением производительности.
\end{frame}
%------------------------------------------------

\begin{frame}
\frametitle{Интерпретация корреляционного анализа}
{\bf 2. Графическое представление}\\

{\it Диаграмма рассеяния:}\\
- На диаграмме рассеяния можно визуально оценить характер связи между переменными.\\
- Если точки на диаграмме располагаются вдоль прямой линии, это подтверждает линейную зависимость.\\
- Распределение точек по диаграмме помогает понять, насколько хорошо одна переменная предсказывает другую.
\newline\\
{\it Линия тренда:}\\
- Линия тренда показывает направление и силу связи. Чем лучше линия тренда соответствует точкам данных, тем сильнее корреляция.\\
- Если линия тренда почти горизонтальная, это указывает на слабую корреляцию или её отсутствие.
\end{frame}
%------------------------------------------------

\begin{frame}
\frametitle{Интерпретация корреляционного анализа}
{\bf 3. Корреляция vs. Причинно-следственная связь}\\
Важно помнить, что корреляция не подразумевает причинно-следственной связи. Даже если две переменные сильно коррелируют, это не означает, что одна переменная вызывает изменения в другой. Корреляция просто указывает на наличие взаимосвязи.\\
{\bf Пример:}\\
Высокий коэффициент корреляции между количеством выпитого кофе и уровнем энергии не означает, что кофе непосредственно повышает уровень энергии. Могут быть другие факторы, такие как общие привычки или уровень стресса, которые влияют на оба параметра.
\end{frame}
%------------------------------------------------

\begin{frame}
\frametitle{Интерпретация корреляционного анализа}
{\bf 4. Проверка значимости}\\
Проверка значимости корреляции помогает определить, является ли наблюдаемая корреляция статистически значимой, а не случайной. Это обычно делается с помощью p-значения:\\
\quad $p < 0.05$: Корреляция статистически значима, и есть основание полагать, что связь не является случайной.\\
\quad $p \geq 0.05$: Корреляция может быть случайной, и нет достаточных доказательств значимой связи.
\newline\\
{\bf Заключение}\\
Интерпретация корреляционного анализа требует понимания как количественных (коэффициент корреляции, значимость), так и качественных аспектов (графическое представление). Это позволяет делать обоснованные выводы о взаимосвязях между переменными, а также формировать гипотезы для дальнейшего анализа и исследования.
\end{frame}
%------------------------------------------------

\begin{frame}
\frametitle{Ограничения корреляционного анализа}
Корреляционный анализ предоставляет полезную информацию о взаимосвязи между двумя переменными, но он имеет свои ограничения. Вот основные из них:
\newline\\
{\bf 1. Корреляция не подразумевает причинно-следственную связь}\\
Корреляция показывает, что две переменные связаны, но не указывает на причину этой связи. Даже при наличии сильной корреляции между переменными нельзя утверждать, что одна переменная вызывает изменение другой.\\
{\bf Пример:} Если наблюдается высокая корреляция между количеством выпитого кофе и уровнем энергии, это не означает, что кофе непосредственно повышает уровень энергии. Возможно, есть другие факторы, влияющие на обе переменные, такие как общий уровень стресса или режим сна.
\end{frame}
%------------------------------------------------

\begin{frame}
\frametitle{Ограничения корреляционного анализа}
{\bf 2. Корреляция измеряет только линейную зависимость}\\
Коэффициент корреляции Пирсона измеряет только линейную зависимость между переменными. Если связь между переменными нелинейная, корреляция может быть низкой, даже если существует сильная зависимость.\\
{\bf Пример:} Если переменные связаны квадратичной зависимостью (например, \( Y = X^2 \)), коэффициент корреляции Пирсона может не отражать этого, и значение \( r \) может быть близким к нулю.
\newline\\
{\bf 3. Чувствительность к выбросам}\\Корреляция сильно подвержена влиянию выбросов или аномальных значений. Один или несколько выбросов могут существенно изменить значение коэффициента корреляции.\\
{\bf Пример:} Если в наборе данных есть несколько точек, которые значительно отклоняются от основной массы данных, это может искажать результаты корреляционного анализа и приводить к неверным выводам.
\end{frame}
%------------------------------------------------

\begin{frame}
\frametitle{Ограничения корреляционного анализа}
{\bf 4. Корреляция может быть случайной}\\
При небольших объемах данных корреляция может быть случайной. Небольшие выборки могут показывать корреляцию, которая в реальности отсутствует.\\
{\bf Пример:} При анализе данных, собранных из небольшой группы людей, можно обнаружить корреляцию, которая не будет наблюдаться в более крупной выборке или в других исследованиях.
\newline\\
{\bf 5. Не учитывает другие переменные}\\
Корреляция не учитывает влияние других переменных. Если есть третья переменная, которая влияет на обе изучаемые переменные, это может исказить результаты корреляционного анализа.\\
{\bf Пример:} Если исследуется связь между количеством часов, проведенных за изучением, и оценками на экзамене, не учитывая, что студенты могут также различаться по уровню мотивации или предыдущему опыту, результат может быть неполным.
\end{frame}
%------------------------------------------------

\begin{frame}
\frametitle{Ограничения корреляционного анализа}
{\bf 6. Не показывает характер зависимости}\\
Коэффициент корреляции не показывает, насколько сильна зависимость между переменными. Например, корреляция 0.8 и 0.9 могут указывать на сильную зависимость, но разница в значении не всегда отражает практическое значение.\\
{\bf Пример:} Хотя корреляция 0.8 может указывать на сильную связь, интерпретация и оценка того, насколько эта связь значима, требует дополнительного анализа и контекста.
\newline\\
{\bf Заключение}\\
Корреляционный анализ — это мощный инструмент для изучения взаимосвязей между переменными, но его ограничения требуют внимательного подхода при интерпретации результатов. Важно помнить, что корреляция не заменяет более глубокий анализ причинно-следственных связей и не учитывает все возможные факторы, влияющие на переменные. Использование корреляционного анализа в сочетании с другими методами, такими как регрессионный анализ и контроль дополнительных переменных, может помочь получить более полное представление о данных.
\end{frame}
\end{document}